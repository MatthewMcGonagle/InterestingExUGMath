\subsection{Quadrature of the Hyperbola and Logarithms}

Now we investigate the relationship between the area under the graph of the hyperbola \(y = 1 / x\) and 
logarithms.

\subsubsection*{The Setup}

John Napier created the logarithm with the purpose of aiding in computations of multiplication and division. \cite{cajori} 
He noted that comparing an arithmetic progression to a geometric progression would allow one to, e.g., switch from
division to subtraction. By having a logarithm table, one can make division easier by converting to the
logarithm using the table, performing subtraction, and then using the logarithm table to convert back. 

So the original intention of logarithms is to take advantage of the algebraic fact that 
\begin{equation}
\log(ab) = \log(a) + \log(b).
\end{equation}

The relationship between logarithms and the area under the hyperbola \(y = 1/x\) was established by the work
of Gregory St. Vincent in 1647 and Alfons Anton de Sarasa in 1649 (note that their work is before the 
invention of calculus of Newton and Leibniz). Using modern terminology, we seek to show that 
\begin{equation}
f(x) \equiv \int\limits_1^x \frac{1}{t} dt,
\end{equation}
satisfies the algebraic rule of algorithms \(f(ab) = f(a) + f(b)\). Since this work predates the invention
of calculus, we will prove this result directly using methods of exhaustion, a technique that also predates
the invention of calculus.

\subsubsection*{The Problem}

\begin{enumerate}
\item Use a method of exhaustion to show that for any \(a, b, c > 0\), one has that
\begin{equation}
\int\limits_{ac}^{bc} \frac{1}{t} dt = \int\limits_a^b \frac{1}{t} dt.
\end{equation} 

\item Prove that
\begin{equation}
\int\limits_1^{ab} \frac{1}{t} dt = \int\limits_1^a \frac{1}{t} dt + \int\limits_1^b \frac{1}{t} dt.
\end{equation}
\end{enumerate}

\subsubsection*{The Solution}

\begin{enumerate}
\item First we partition \([ac, bc]\) using the points \(t_i = ac + \frac{bc - ac}{N} i \), for \(0 \leq i \leq N\).
Next, note that the points \(s_i = a + \frac{b - a}{N} i\) for \(0 \leq i \leq N\) give a partition of
the interval \([a, b]\).

Let \(U_N\) be the upper sum for the partition of \([ac, bc]\). We have that
\begin{align}
U_N & = \sum\limits_{i = 0}^{N-1} \frac{1}{t_i} \frac{bc - ac}{N}, \\
    & = \sum\limits_{i = 0}^{N - 1} \frac{1}{a + \frac{b - a}{N}i} \frac{b - a}{N}, \\
    & = \sum\limits_{i = 0}^{N - 1} \frac{1}{s_i} \frac{b - a}{N}, \\
\end{align}
Let us relate this to \(\int_a^b 1/t dt\). Now, we see that \(U_N\) isn't quite a lower bound for this integral.
However, we do have that it gives a lower bound for a slightly different interval. Let 
\(\delta_N = \frac{b - a}{N}\), we have that
\begin{equation}
U_N < \int\limits_{a - \delta_N}^{b - \delta_N} \frac{1}{t} dt.
\end{equation}

Similarly let the lower sum be \(L_N\); we have that 
\begin{align}
L_N & = \sum\limits_{i = 0}^{N-1} \frac{1}{t_{i+1}} \frac{bc - ac}{N}, \\ 
    & = \sum\limits_{i = 0}^{N - 1} \frac{1}{a + \frac{b - a}{N}(i+1)} \frac{b - a}{N}, \\
    & = \sum\limits_{i = 0}^{N - 1} \frac{1}{s_{i + 1}} \frac{b - a}{N}, 
\end{align}
We also see that
\begin{equation}
L_N > \int\limits_{a + \delta_N}^{b + \delta_N} \frac{1}{t} dt.
\end{equation}

So we have that
\begin{equation}
\int\limits_{a + \delta_N}^{b + \delta_N} \frac{1}{t} dt < \int\limits_{ac}^{bc} \frac{1}{t} dt
<  \int\limits_{a - \delta_N}^{b - \delta_N} \frac{1}{t} dt.
\end{equation}

As we let \(N \to \infty\), we have that \(\delta_N \to 0\). Therefore, we get that
\begin{equation}
\int\limits_a^b \frac{1}{t} dt \leq \int\limits_{ac}^{bc} \frac{1}{t} dt
    \leq \int\limits_a^b \frac{1}{t} dt.
\end{equation}
Therefore,
\begin{equation}
\int\limits_{ac}^{bc} \frac{1}{t} dt = \int\limits_a^b \frac{1}{t} dt.
\end{equation}

\item Next, consider the area
\begin{equation}
\int\limits_1^{ab} \frac{1}{t} dt.
\end{equation} 
From our previous result, we have that
\begin{equation}
\int\limits_{a}^{ab} \frac{1}{t} dt = \int\limits_{1}{b} \frac{1}{t} dt.
\end{equation}

Next, note that 
\begin{equation}
\int\limits_a^{ab} \frac{1}{t} dt = \int\limits_1^{ab} \frac{1}{t} dt - \int\limits_1^a \frac{1}{t} dt.
\end{equation}
Therefore, we get that
\begin{equation}
\int\limits_1^{ab} \frac{1}{t} dt = \int\limits_1^a \frac{1}{t} dt + \int\limits_1^b \frac{1}{t} dt.
\end{equation}

\end{enumerate}
