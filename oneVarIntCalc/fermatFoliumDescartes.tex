\subsection{Fermat's Quadrature of the Folium of Descartes}

\subsubsection*{The Setup}

In this section we investigate how Fermat was able to use integration by parts to find the area of the
loop in the Folium of Descartes. A reference for Fermat's treatment of this problem is given by \cite{fermatTreatise}. 

The Folium of Descartes is the curve given by
\begin{equation}
x^3 + y^3 = 3C xy,
\end{equation}
where \(C\) is a constant. A graph of the curve for \(C = 1\) is given below. The curve crosses itself
at the origin.

\begin{figure}[h]
\centering
\includegraphics[width=3in]{_generated/graphFoliumDescartes.pdf}
\end{figure}

Let \(\bar x\) be the x-coordinate of the right-most part of the loop; so the loop can be written as the graph
of two functions \(y_1 \leq y_2\) on \([0, \bar x]\).

Fermat's idea comes in two parts:
\begin{enumerate}
\item If we can make a substitution for a new variable \(z\) to replace \(y\) given by \(y = C^{1 - m - n} x^m z^n\) such that the equation
now becomes
\begin{equation}
x^p = \sum\limits_{i} D_i^{p - p_i} z^{p_i},
\end{equation}
then we know how to compute the integral
\begin{equation}
\int\limits x^p dz.
\end{equation}
A note here about the choice of \(C^{1 - m - n}\), Fermat likes to follow the rules of homogeneity, which is probably best explained
in terms of units. If \(x\) and \(y\) were distances in say meters, then the left hand side of \(x^3 + y^3 = 3Cxy\) has units 
\(\text{meters}^3\). Fermat chooses a constant \(C\) that is also in meters, and so the right hand side also has units
\(\text{meters}^3\). Similarly, the left and right hand sides of \(y = C^{1 - m -n} x^m z^n\) have units of just meters. 

\item Use integration by parts to write the original area integral in terms of the integral we now know how to compute. Let \(\bar z\) be
the largest value of the \(z\)-coordinate obtained by the loop (we will in fact see that \(\bar z = \infty\)). So we wish to try to find 
\(p\) and \(p_i\) to ensure that
\begin{equation}
\int\limits_0^{\bar x} (y_2 - y_1) dx = \int\limits_0^{\bar z} x^p dz.
\end{equation}  
It may be that \(\bar z = \pm \infty\), and so to be rigorous we will need to take an improper integral.
\end{enumerate}

\subsubsection*{The Problem}

Do the two steps of Fermat's plan to evaluate the area of the loop in the Folium of Descartes.

\subsubsection*{The Solution}

\begin{enumerate}
\item Let's do step one; let's make the substitution \(y = C^{1 - m - n} x^m z^n\) into \(x^3 + y^3 = 3Cxy\). We get
\begin{equation}
x^3 + C^{3 - 3m - 3n} x^{3m} z^{3n} = 3C^{2 - m - n} x^{1 + m} z^n.
\end{equation}
Let's us now try to consolidate the \(x\)-terms into one term \(x^p\); to do so, two of the terms need to have the same power of \(x\). 
Let's look at the three cases:
    \begin{enumerate}
    \item If we try equating \(x^3 = x^{3m}\), then we must have \(m = 1\). So we get
    \begin{equation}
    1 + C^{-3n} z^{3n} = 3C^{1 - n} x^{-1} z^n.
    \end{equation} 
    Consolidating the z-terms, this becomes
    \begin{equation}
    3C^{1 - n} x^{-1} = z^{-n} + C^{-3n} z^{2n}.
    \end{equation}
    The problem here is that there is no way to produce the power \(x^{-1}\) using integration by parts without using a logarithm. So
    this case isn't really any help.

    \item Now let's try equating \(x^{3m} = x^{1 + m}\). We then have that \(m = \frac{1}{2}\). So we get
    \begin{equation}
    x^{3/2} = -C^{3/2 - 3n} z^{3n} + 3 C^{3/2 - n} z^n.
    \end{equation}   
    Again, we can't directly get \(x^{3/2}\) without introducing some radical power of \(x\) that isn't already present. We could just
    square the equation, but we will see that it is better to work with the last case.

    \item Now we equate \(x^3 = x^{1 + m}\) and we get \(m = 2\). So then we have that
    \begin{equation}
    1 + C^{-3 - 3n} x^3 z^{3n} = 3C^{-n} z^n,
    \end{equation}
    which gives us
    \begin{equation}
    x^3 = 3C^{3 + 2n} z^{-2n} -C^{3 + 3n} z^{-3n}. 
    \end{equation}
    The power \(x^3\) is a nice integer power for us to work with.
    \end{enumerate}

So we take \(y = C^{-1 - n} x^2 z^n\) where \(n\) is still left to be determined. Our equation defining the folium is now
\begin{equation}
x^3 = 3C^{3 + 2n} z^{-2n} - C^{3 + 3n} z^{-3n}.
\end{equation}

\item
Let us now consider the area integral
\begin{equation}
\int\limits_0^a (y_2 - y_1) dx = C^{-1 - n} \int\limits_0^a (z_2^n - z_1^n) x^2 dx
\end{equation}
When we perform integration by parts, we will get an integral of the form (ignoring the boundary terms for now)
\begin{equation}
\frac{n}{3}\int z^{n - 1} x^3 dz. 
\end{equation}
This is most simple when \(n = 1\). So we will take \(y = C^{-2} x^2 z\) for our variable transformation. The equation of the folium is now
\begin{equation}
x^3 = 3C^5 z^{-2} - C^6 z^{-3}.
\end{equation}
Below is a picture of the Folium of Descartes in \(xz\)-coordinates.
\begin{figure}[h]
\centering
\includegraphics[width = 3in]{_generated/xzGraphFolium.pdf}
\end{figure}

Now we need to be more rigorous with our boundary terms. 

First note that the loop of the folium of descartes is in the first quadrant, i.e. \(\{x,y \geq 0\}\). 
Furthermore, the only non-negative solutions to \(3C^5 z^{-2} - C^6 z^{-3} = 0 \) are \(z = \frac{C}{3}\) and \(z = \infty\). 

Next, note that for \(x^3 + y^3 = 3Cxy\) and \(x, y\) close to the origin, the \(3Cxy\) term dominates. So we expect the loop to
be asymptotically like \(3xy = 0\). That is, one direction of approach to the origin is horizontal while the other is vertical. We need
to investigate this asymptoticness more closely.

For any point \((x,y)\) on the folium and close to the origin, let \(\lambda = \|(x,y)\|\) and let \((\tilde x, \tilde y) = \frac{1}{\lambda}(x, y)\). 
Using the equation for the folium, we have that 
\begin{equation}
0 \leq \tilde x \tilde y \leq \frac{\lambda}{3C}.  
\end{equation}
Note that \((\tilde x, \tilde y)\) occur on the unit circle. Consider the function \(f(x,y) = xy\) and its parameterization on the unit circle
given by \(g(\theta) = f(\cos\theta, \sin\theta) = \cos\theta \sin\theta\). Now note that where \(g\) vanishes for the first quadrant is \(\theta = 0\) or
\(\theta = \pi/2\). Furthermore, \(g'(0)\neq 0\) and \(g'(\pi/2) \neq 0\). 

We can use this and some basic trignometry to show that \(\tilde x \tilde y \leq \frac{\lambda}{3C}\) gives us that \(\|(\tilde x, \tilde y) - (1,0)\| \leq K \lambda\) or
\(\|(\tilde x, \tilde y) - (0, 1)\| \leq K \lambda\) for some constant \(K\) depending on \(C\).

The first case is the horizontal part of the loop approaching the origin. In this case, we have that \(z = C^2 y / x^2 = C^2 \frac{\tilde y} {\lambda \tilde x^2}\). Now, as we approach
the origin, \(\lambda \to 0\), \(\tilde x \to 1\) and \(\tilde y \leq K \lambda\). So we get that \(z\) is bounded as we approach the origin along the horizontal portion of the loop.
So it must approach \(z = \frac{C}{3}\).

The second case is the vertical portion of the loop approacing the origin. In this case, we write \(z = C^2 \frac{\tilde y}{x\tilde x}\). We have that \(\tilde x \to 0\), \(x \to 0\), and
\(\tilde y \to 1\). Therefore \(z \to \infty\) along the vertical portion of the loop as we approach the origin.

From the \(xz\)-equation of the folium, we see that in either case that \(zx^3 \to 0\) as \(x \to 0\). Now let us rigorously apply integration by parts.
Let \(z_a\) be the z-coordinate of the right-most part of the loop.
We have that
\begin{equation}
\int\limits_0^{\bar x} (z_2 - z_1) x^2 dx = \lim\limits_{b \to 0+} \int\limits_b^{\bar x} (z_2 - z_1) d \left(\frac{x^3}{3}\right).
\end{equation}
Let us study each integral involving \(z_i\) individually. Let \(z_a\) be the z-coordinate of the rightmost part of the loop. For the bottom part of the loop, we see that
\begin{align}
\lim\limits_{b \to 0+} \int\limits_b^{\bar x} z_1 d\left(\frac{x^3}{3}\right) & =  \lim\limits_{b \to 0+} z_1\frac{x^3}{3} \big|_b^a -\frac{1}{3} \int\limits_{C/3}^{z_a} x^3 dz, \\
    & = \frac{z_b a^3}{3} - 0 - \frac{1}{3} \int\limits_{C/3}^{z_a} x^3 dz. 
\end{align}
For the top part of the loop, we see that
\begin{align}
\lim\limits_{b \to 0+} \int\limits_b^{\bar x} z_2 d\left(\frac{x^3}{3}\right) & = \lim\limits_{b \to 0+} z_2\frac{x^3}{3} \big|_b^a - \frac{1}{3} \int\limits_{\infty}^{z_a} x^3 dz, \\
    & = \frac{z_a a^3}{3} + \frac{1}{3} \int\limits_{z_a}^\infty x^3 dz. 
\end{align}
Putting them together, we get
\begin{align}
\lim\limits_{b \to 0+} \int\limits_b^{\bar x} (z_2 - z_1) d \left(\frac{x^3}{3}\right) & =  \frac{1}{3} \int\limits_{z_a}^\infty x^3 dz + \frac{1}{3} \int\limits_{C/3}^{z_a} x^3 dz, \\ 
    & = \frac{1}{3} \int\limits_{C/3}^\infty x^3 dz. 
\end{align}
Now, we can use the \(xz\)-coordinate equation for the folium to get that
\begin{align}
\frac{1}{3} \int\limits_{C/3}^\infty x^3 dz & = \int\limits_{C/3}^\infty \left( C^5 z^{-2} - \frac{C^6 z^{-3}}{3} \right) dz, \\
    & = C^5 \frac{3}{C} - C^6\frac{9}{6C^2}, \\
    & = 3C^4 - \frac{3C^4}{2}, \\
    & = \frac{3C^4}{2}. 
\end{align}

Combining with the area integral, we get
\begin{equation}
\int\limits_0^{\bar x} (y_2 - y_1) dx = \frac{3}{2}C^2.
\end{equation}
\end{enumerate}
