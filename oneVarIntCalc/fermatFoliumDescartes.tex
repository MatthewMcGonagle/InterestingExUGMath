\subsection{Fermat's Quadrature of the Folium of Descartes}

\subsubsection*{The Setup}

In this section we investigate how Fermat was able to use integration by parts to find the area of the
loop in the Folium of Descartes. A reference for Fermat's treatment of this problem is given by \cite{fermatTreatise}. 

The Folium of Descartes is the curve given by
\begin{equation}
x^3 + y^3 = 3C xy,
\end{equation}
where \(C\) is a constant. A graph of the curve for \(C = 1\) is given below. The curve crosses itself
at the origin.

\begin{figure}[h]
\centering
\includegraphics[width=3in]{oneVarIntCalc/graphFoliumDescartes.pdf}
\end{figure}

Fermat's idea comes in two parts:
\begin{enumerate}
\item If we can make a substitution for a new variable \(z\) to replace \(y\) given by \(y = C^{1 - m - n} x^m z^n\) such that the equation
now becomes
\begin{equation}
x^p = \sum\limits_{i} D_i^{p - p_i} z^{p_i},
\end{equation}
then we know how to compute the integral
\begin{equation}
\int\limits x^p dz.
\end{equation}
A note here about the choice of \(C^{1 - m - n}\), Fermat likes to follow the rules of homogeneity, which is probably best explained
in terms of units. If \(x\) and \(y\) were distances in say meters, then the left hand side of \(x^3 + y^3 = 3Cxy\) has units 
\(\text{meters}^3\). Fermat chooses a constant \(C\) that is also in meters, and so the right hand side also has units
\(\text{meters}^3\). Similarly, the left and right hand sides of \(y = C^{1 - m -n} x^m z^n\) have units of just meters. 

\item Use integration by parts to write the original area integral in terms of the integral we now know how to compute. That is, 
we try to find \(p\) and \(p_i\) to ensure that
\begin{equation}
\int\limits_0^{\bar x} (y_2 - y_1) dx = \int\limits_0^{\bar z} x^p dz.
\end{equation}  
It may be that \(\bar z = \pm \infty\), and so to be rigorous we will need to take an improper integral.
\end{enumerate}

\subsubsection*{The Problem}

Do the two steps of Fermat's plan to evaluate the area of the loop in the Folium of Descartes.

\subsubsection*{The Solution}

\begin{enumerate}
\item Let's do step one; let's make the substitution \(y = C^{1 - m - n} x^m z^n\) into \(x^3 + y^3 = 3Cxy\). We get
\begin{equation}
x^3 + C^{3 - 3m - 3n} x^{3m} z^{3n} = 3C^{2 - m - n} x^{1 + m} z^n.
\end{equation}
Let's us now try to consolidate the \(x\)-terms into one term \(x^p\); to do so, two of the terms need to have the same power of \(x\). 
Let's look at the three cases:
    \begin{enumerate}
    \item If we try equating \(x^3 = x^{3m}\), then we must have \(m = 1\). So we get
    \begin{equation}
    1 + C^{-3n} z^{3n} = 3C^{1 - n} x^{-1} z^n.
    \end{equation} 
    Consolidating the z-terms, this becomes
    \begin{equation}
    3C^{1 - n} x^{-1} = z^{-n} + C^{-3n} z^{2n}.
    \end{equation}
    The problem here is that there is no way to produce the power \(x^{-1}\) using integration by parts without using a logarithm. So
    this case isn't really any help.

    \item Now let's try equating \(x^3m = x^{1 + m}\). We then have that \(m = \frac{1}{2}\). So we get
    \begin{equation}
    x^{3/2} = -C^{3/2 - 3n} z^{3n} + 3 C^{3/2 - n} z^n.
    \end{equation}   
    Again, we can't directly get \(x^{3/2}\) without introducing some radical power of \(x\) that isn't already present. We could just
    square the equation, but we will see that it is better to work with the last case.

    \item Now we equate \(x^3 = x^{1 + m}\) and we get \(m = 2\). So then we have that
    \begin{equation}
    1 + C^{-3 - 3n} x^3 z^{3n} = 3C^{-n} z^n,
    \end{equation}
    which gives us
    \begin{equation}
    x^3 = 3C^{3 + 2n} z^{-2n} -C^{3 + 3n} z^{-3n}. 
    \end{equation}
    The power \(x^3\) is a nice integer power for us to work with.
    \end{enumerate}

So we take \(y = C^{-1 - n} x^2 z^n\) where \(n\) is still left to be determined. Our equation defining the folium is now
\begin{equation}
x^3 = 3C^{3 + 2n} z^{-2n} - C^{3 + 3n} z^{-3n}.
\end{equation}

\item
Let us now consider the area integral
\begin{equation}
\int\limits_0^a (y_2 - y_1) dx = C^{-1 - n} \int\limits_0^a (z_2^n - z_1^n) x^2 dx
\end{equation}
When we perform integration by parts, we will get an integral of the form (ignoring the boundary terms for now)
\begin{equation}
\frac{n}{3}\int z^{n - 1} x^3 dz. 
\end{equation}
This is most simple when \(n = 1\). So we will take \(y = C^{-2} x^2 z\) for our variable transformation. The equation of the folium is now
\begin{equation}
x^3 = 3C^5 z^{-2} - C^6 z^{-3}.
\end{equation}
Now we need to be more rigorous with our boundary terms.
\end{enumerate}
