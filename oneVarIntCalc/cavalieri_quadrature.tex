\subsection{Cavalieri's Quadrature of \(x^n\)}

In this example we consider Cavalieri's method of finding the area under \(x^n\) for positive integers \(n\). In
particular, we will use a couple simple integral properties, which may be simply proven using a 
method of exhaustion, to create a system of linear equations that will allow us to compute 
\begin{equation}
\int\limits_0^a x^n dx,
\end{equation}
for \(a > 0\) and any positive integer \(n\). Actually we will create an algorithm for finding the value of such an
integral for a given value of \(n\) (\(a\) will still be general). The author is unaware of how 
mathematical induction may be applied to find the general formula for general \(n\) and general \(a\); the
equations seem to be too complicated to be immediately tractable.

\subsubsection*{The Setup}

First, an introduction to Cavalieri's idea. Struik \cite{struik} gives a nice discussion of Cavalieri's
methods with more modern notation for \(n = 2, 3\).
Cavalieri uses a notion of "summing over all lines", "summing over all squares,", "summing over all cubes", etc;
for example, he is comfortable with writing in words the equivalent of \(\sum x^2\), where the sum is considered to
be over all x-values for \(0 \leq x \leq a\). He does not consider details of convergence or whether such a sum
is well-defined. This method is very non-rigorous by modern standards, and we will
update his arguments; when necessary, we will prove the integral properties we need using an exhaustion by Riemann
sums. 

Furthermore, the argument in Struik \cite{struik} doesn't precisely match ours, but the author feels our argument
amounts to an induction step that is easier to analyze. The difference being that we will induct over set of
integrals depending on \(a\) and \(n\), instead of just the values of \(\int_0^a x^n dx\). Next we introduce
the set of integrals we need for our induction.

We introduce another variable \(y\) such that \(x + y = a\); here Cavalieri is imagining
breaking the line segment of length \(a\) into two parts. Then the variables \(x\) and \(y\) represent their
lengths, which must add up to \(a\). We will be interested in the following integrals
\begin{equation}
I^n_{p, q} \coloneqq \int\limits_0^a x^p y^q dx,
\end{equation}
where \(p, q\) are non-negative integers such that \(p + q = n\). Note that our integrals may
be rewritten entirely in terms of \(x\) as \(I^n_{p, q} = \int_0^a x^p (a - x)^q dx\), but we will find
it more convenient to express them in terms of \(x\) and \(y\). 

Next, we describe our induction step. We will assume that we know that values of \(I^m_{p, q}\) for all
\(m < n\) and \(p, q\) non-negative integers with \(p + q = m\). Then we will show that we can compute
\(I^n_{p, q}\) for all \(p, q\) non-negative integers with \(p + q = n\).

Cavalieri's idea can be broken into several parts:
\begin{enumerate}
\item There is a symmetry coming from interchanging the roles of \(x\) and \(y\), i.e. \(I^n_{p, q} = I^n_{q, p}\). 
This implies that there are actually less than \(n+1\) unknowns. They can be expressed in the form
\(I^n_{p, q}\) for \(p, q\) non-negative integers with \(p + q = n\).

In the case of even \(n = 2k\), we only have the
\(k+1\) unknowns \(I^{2k}_{0, 2k}, ..., I^{2k}_{k, k}\). In the case of odd \(n = 2k + 1\), we have only have the
\(k+1\) unknowns \(I^{2k+1}_{0, 2k+1}, ..., I^{2k+1}_{k, k+1}\).

The symmetry may be proven using a method of exhaustion. Take a uniform partition of \([0, a]\) with an even
number of intervals \(2P\) of length \(h = a/2P\). Let \(x_i\) be the center of each partition interval.
We approximate the integral using the sum
\begin{equation}
J^n_{p, q, P} \coloneqq \sum\limits_1^{2P} x_i^p y_i^q h,
\end{equation}
and we have that \(I^n_{p, q}  = \lim\limits_{P \to \infty} J^n_{p, q, 2P}\). By the symmetry of our partition
and choice of \(x_i\), we have the symmetry \(J^n_{p, q, P} = J^n_{q, p, P}\). Taking the limit we get that
\(I^n_{p, q} = I^n_{q, p}\). 

\item The next part is to create recurrence relations for the \(I^n_{p, q}\) for the induction step using
\begin{align}
a I^{n-1}_{p, q} & = \int\limits_0^a (y + x)x^p y^q dx \\ 
    & = I^{n}_{p, q+1} + I^n_{p+1, q},
\end{align} 
where \(p + q = n-1\).

We see that for the case of even \(n = 2k\), we have the \(k\) equations in the \(k+1\) unknowns:
\begin{align}
aI^{2k-1}_{0, 2k-1} & = I^{2k}_{0, 2k} + I^{2k}_{1, 2k-1}, \\
... & \\
aI^{2k-1}_{k-2, k+1} & = I^{2k}_{k-2, k+2} + I^{2k}_{k-1, k+1}, \\
aI^{2k-1}_{k-1, k} & = I^{2k}_{k-1, k+1} + I^{2k}_{k, k}. 
\end{align}
Note that this is an underdetermined system, and we will need another equation. In the next part we will
obtain another equation for this case.

For the case of odd \(n = 2k + 1\), we have the \(k+1\) equations in the \(k+1\) unknowns:
\begin{align}
a I^{2k}_{0, 2k} & = I^{2k+1}_{0, 2k+1} + I^{2k+1}_{1, 2k}, \\
... \\
a I^{2k}_{k-1, k+1} & = I^{2k+1}_{k-1, k+2} + I^{2k+1}_{k, k+1}, \\
a I^{2k}_{k, k} & = 2 I^{2k+1}_{k, k+1}.
\end{align}
We can see that this system is actually solvable; the last equation allows us to solve for \(I^{2k+1}_{k, k+1}\),
and then we work backward substitute in equations to solve for the rest of the quantities one by one. 

\item For the final part we need to find another equation for the case of even \(n = 2k\). Cavalieri's idea
involves a scaling argument to find a relation between \(I^{2k}_{0, 2k}\) and \(I^{2k}_{k, k}\). 

First, introduce a variable \(z\) such that \(x = a/2 + z\) and \(y = a/2 - z\); that is, \(z\) represents the
offset of \(x\) from \(a/2\).

Then we have that
\begin{equation}
I^{2k}_{k, k} = \int\limits_0^a x^k y^k dx = \int\limits_0^a (a^2/4 - z^2)^k dx.
\end{equation}
An exhaustion argument can be used to easily show \(\int_0^a (f \pm g) dx = \int_0^a f dx \pm \int_0^a g dx\),
especially in the case that \(f\) is non-negative and \(g\) is non-positive.

Then, we have that
\begin{equation}
I^{2k}_{k, k} = 
    \sum\limits_{l = 0}^{k} {k \choose l} (-1)^l (a/2)^{2l} \int\limits_0^a z^{2(k-l)} dx. 
\end{equation}

Now we claim that \(J^{2p} \coloneqq \int_0^a z^{2p} dx = (1/2)^{2p} \int_0^a x^{2p} dx\); this can be proven
by breaking the integral into pieces and applying a scaling argument. 

First note that
\begin{align}
J^{2p} & = \int_0^{a/2} z^{2p} dx + \int_{a/2}^a z^{2p} dx, \\ 
    & \coloneqq J^{2p}_1 + J^{2p}_2
\end{align}
One can use an exhaustion argument to easily show that an integral is translation invariant. Next,
not that \(z = x - a/2\), so that \(z\) is the translation of \(x\) by \(a/2\). Therefore, we have
that \(J^{2p}_2 = \int_0^{a/2} x^{2p} dx\).

Similarly we have that \(J^{2p}_1 = \int_{-a/2}^0 x^{2p} dx\); since \(x^{2p}\) is an even power, then it 
is also an even function. Therefore \(J^{2p}_1 = \int_0^{a/2} x^{2p} dx\).

Therefore we have that \(J^{2p} = 2 \int_0^{a/2} x^{2p} dx\).

Next we claim that \(\int_0^{a/2} x^{2p} dx = (1/2)^{2p+1}\int_0^a x^p dx\). To prove the we apply a scaling
argument to an exhaustion of \(\int_0^{a/2}x^{2p} dx\). We partition the interval \([0, a/2]\) into \(Q\)
uniform intervals of length \(h_Q = a / 2Q\) and \(x_i\) be the left endpoints of each interval. 

Then we have that \(J^{2p}_Q \coloneqq \sum\limits_i x_i^{2p} h_Q\). Next note that \(x_i = (1/2) \tilde x_i\)
for points \(\tilde x_i\) of a partition of \([0, a]\). Similarly \(h_{2Q} = (1/2) \tilde h_Q\), where
\(\tilde h_Q\) is the length of
the corresponding uniform partition of \([0, a]\). Therefore, we have that 
\(J^{2p}_Q = (1/2)^{2p + 1} \sum\limits_i \tilde x_i^{2p} \tilde h_Q\), a scalar multiple of an exhaustion for
\(\int_0^a x^{2p} dx\). Therefore we get that
\begin{align}
\int\limits_0^a z^{2p} dx & = (1/2)^{2p} \int\limits_0^a x^{2p} dx, \\
    & = (1/2)^{2p} I^{2p}_{2p, 0}.
\end{align}
So we have that
\begin{equation}
I^{2k}_{k, k} = \sum\limits_{l=0}^k {k \choose l} (-1)^l (a/2)^{2l} (1/2)^{2(k-l)} I^{2(k-l)}_{2(k-l), 0},
\end{equation}
and we get the equation
\begin{equation}
(1/2)^{2k} \sum\limits_{l=1}^k {k \choose l} (-1)^{l+1} a^{2l} I^{2(k-l)}_{2(k-l), 0} = 
    (1/2)^{2k} I^{2k}_{2k, 0} - I^{2k}_{k, k},
\end{equation}
where the left hand side is known. It is possible to see that after including this system, we get \(k+1\) equations
in the \(k+1\) unknowns system for \(I^{2k}_{0, 2k}, ..., I^{2k}_{k,k}\), and this system is in fact 
solvable (it is easier this last fact using the matrix structure). 
\end{enumerate}

Note, you may be tempted to introduce an equation for \(I^{2k}_{k, k}\) by using that
\(a^{2k+1} = \int_0^a (x + y)^{2k} dx\) and binomial expanding. However, this equation is a consequence
of the \(k\) equations we already found. Namely we have that
 \(a^{2k+1} = a \int_0^a(x+y)^{2k-1}dx = a\sum_{p=0}^{2k-1} {2k-1 \choose p} I^{2k-1}_{p, 2k-1-p}\), and
then we can use that
\({2k \choose p} = {2k-1 \choose p} + {2k-1 \choose p-1}\) to relate the two binomial expansions in each equation. 
So trying to use the binomial expansion will not give us anything new. 

It is noteworthy that amazingly we may solve for the case of odd \(n\) by just solving a linear system that
it created from very simple knowledge of integrals (i.e. no need to do anything complicated like the \(z\) trick
used in the even case).

Let us demonstrate how to use Cavalieri's ideas to compute the cases of \(n = 2, 3\). 

\subsubsection*{Example for the Case of \(n = 2\)}

First note that \(\int_0^a x dx  = \int_0^a y dx = a^2 / 2\) follows from basic geometry. So we already know the
base case of \(n = 1\). Our unknowns are \(I^2_{0, 2}\) and \(I^2_{1, 1}\). 

We start by writing down the recurrence relation
\begin{equation}
a^3/2 = I^2_{0, 2} + I^2_{1, 1}. 
\end{equation}
Next we expand
\begin{align}
I^2_{1, 1} & = \int_0^a (a^2 /4 - z^2) dx, \\
    & = a^3/4 - (1/4) I^2_{0, 2}. 
\end{align}
We can solve these equations to get
\begin{align}
I^2_{0, 2} & = a^3 / 3, \\
I^2_{1, 1} & = a^3 / 6. 
\end{align}

\subsubsection*{Example for the Case of \(n = 3\)}

We start by writing the recurrence relations
\begin{align}
a I^2_{0, 2} = a^4/3 & = I^3_{0, 3} + I^3_{1, 2}, \\
a I^2_{1, 1} = a^4 / 6 & = 2 I^3_{1, 2}. 
\end{align}

We see that we can directly solve the last equation for \(I^3_{1, 2} = a^4 / 12\), and then we can
solve the first equation for \(I^3_{0, 3} = a^4 / 4\).

\subsubsection*{The Problem}
Use Cavalieri's ideas to compute the next two cases of \(n = 4, 5\).

\subsubsection*{The Solution}

Let us first compute the case of \(n = 4\). We need to compute the integrals \(I^4_{0, 4}, I^4_{1, 3}, I^4_{2, 2}\).
We start by writing down the recurrence relations
\begin{align}
a I^3_{0, 3} = a^5 / 4 & = I^4_{0, 4} + I^4_{1, 3}, \\
a I^3_{1, 2} = a^5 / 12 & = I^4_{1, 3} + I^4_{2, 2}.
\end{align}

Next we expand
\begin{align}
I^4_{2, 2} & = \int\limits_0^a (a^2 / 4 - z^2)^2 dx, \\
    & = a^5 / 16 - (a^2 / 2) \int\limits_0^a z^2 dx + \int\limits_0^a z^4 dx, \\
    & = a^5 / 16 - a^5 / 24 + (1/16)I^4_{0, 4}, \\
    & = a^5 / 48 + (1/16)I^4_{0, 4}.
\end{align}
So we have three equations in three unknowns, and this system is in fact solvable. We have \(I^4_{0, 4} = a^5/5\),
\(I^4_{1, 3} = a^5 / 20\), and \(I^4_{2, 2} = a^5 / 30\).

Next we consider the case of \(n = 5\). We still have only three unknowns: \(I^5_{0, 5}, I^5_{1, 4},\) and
\(I^5_{2, 3}\). Let us write down the recurrence relations:
\begin{align}
a I^4_{0, 4} = a^6 / 5 = I^5_{0, 5} + I^5_{1, 4}, \\
a I^4_{1, 3} = a^6 / 20 = I^5_{1, 4} + I^5_{2, 3}, \\
a I^4_{2, 2} = a^6 / 30 = 2 I^5{2, 3}. 
\end{align} 
We can solve this system for our unknown integrals (no use of \(z\) necesssary). We obtain
\(I^5_{0, 5} = a^6 / 6\), \(I^5_{1, 4} = a^6 / 30\), and \(I^5_{2, 3} = a^6 / 60\).

