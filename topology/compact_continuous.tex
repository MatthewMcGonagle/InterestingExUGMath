\subsection{A Nonmetrizable Topology for Compactly Supported Continuous Functions}

In this section we look at a topology on the compactly supported continuous functions on \(\mathbb R\), 
i.e. \(C_0(\mathbb R) = \{f \in C(\mathbb R) | \text{suppt}(f) \text{ compact}\}\). 

\subsection*{The Setup}

We will define the topology using an exhaustion of \(\mathbb R\) by compact intervals. So first
define 
\begin{equation}
U_n = \{f \in C_0(\mathbb R) | \text{suppt}(f) \subset [-n, n]\}.
\end{equation}
Note that \(C_0(\mathbb R) = \bigcup_n U_n\). Each \(U_n\) is given the topology of uniform convergence, 
i.e. the open sets of \(U_n\) are generated by the balls 
\(B_\epsilon(f) = \{g \in  U_n | \sup\limits_{x\in [-n, n]} |f(x) - g(x)| < \epsilon\}\) for \(f \in U_n\).

We define a topology \(\tau\) on \(C_0(\mathbb R)\) using the final topology given by the inclusions
\(U_n \to C_0(\mathbb R)\). In particular, \(U \subset C_0(\mathbb R)\) is open if and 
only if \(U \bigcap U_n\) is open for all \(n\).

This topology is a useful topology for the theory of distributions \cite{KrantzParks}(page 180).
In particular, it has many continuous linear functionals. For example, the functional 
\(T : C_0(\mathbb R) \to \mathbb R\) defined by \(T(f) = \int_{\mathbb R} x^2 f(x) dx\) is continous for
the topology \(\tau\), but it is not continuous for the topology of the supremum norm (i.e. the topology
induced by \(\|f\| = \sup |f(x)|\)). 

The topology \(\tau\) can also be described as the largest topology on \(C_0(\mathbb R)\) that
allows the inclusions \(U_n \to C_0(\mathbb R)\) to be continuous.

There are alternative ways to realize this topology, see counterexample 1.1.8 of
\cite{HamiltonInverseFunction}. In particular, it can be seen as coming from an uncountable collection
of semi-norms.

\subsection*{The Problem}

Give a direct proof that the above topology on \(C_0(\mathbb R)\) is non-metrizable. 

\subsection*{The Solution}

Let \(d\) be any metric on \(C_0(\mathbb R)\) such that the topology \(\tau_d\) generated by the metric \(d\) is
contained in \(\tau\). We will show that there exists a sequence \(f_n \to 0\) in \(\tau_d\), but
\(f_n \not \to 0\) in \(\tau\).

First note that for any \(n > 0\), we can find a sequence of non-zero functions \(f_k\) with support
in \([n-1, n+1]\) such \(f_k \to 0\) in \(\tau\) (simply take a sequence uniformly converging 
to \(0\) in \(U_{n + 1}\) with the appropriate support restrictions). In particular, we may take that
\(f_k(n) > 0\) for all \(k\). Therefore, given a ball \(B_{1/n}(0)\)
for the metric \(d\), we can find a function \(f_n\) such that \(f_n(n) > 0\) for all \(n\)
and \(f_n \in B_{1/n}(0)\). We see that \(f_n \to 0\) in \(\tau_d\).

However, it is then easy to construct a function \(\rho\) such that \(0 < \rho(n) < f_n(n)/2\) for all \(n\).
We see that \(V = \{|f(x)| < \rho(x)\}\) is open in \(\tau\), but \(f_n \not \in V\) for all \(n\). Therefore
\(f_n \not \to 0\) in \(\tau\), despite \(f_n \to 0\) in \(\tau_d\). 
