\subsection{Global Analysis}
\subsubsection*{The Setup}

It is sometimes possible for us to use topology to extend an argument that holds in a limit to a more global case.

Here we look at an example from Osserman \cite{Osserman} where we can use winding number arguments to show that
a map is in fact a diffeomorphism. In particular, consider the domain \(D = \{0 < |z| < 1\} \subset \mathbb C\}\), 
and consider the map 
\begin{equation}
f(z) = \frac{1}{\bar z} + \frac{z^3}{3},
\end{equation} 
defined on \(D\). Let \(C = \{|z| = 1\}\) be the outermost component of the boundary of \(D\). 

Let us next show that \(f\) is a local diffeomorphism. Recall that we by the inverse function theorem, we need to 
only show that on all points of \(D\) we have that \(Df\) is of rank two. This is equivalent to only having
trivial solutions \(r_i \in \mathbb R\) to \(r_1\partial_x f + r_2\partial_y f\); this is equivalent to only
a trivial solution \(c\in \mathbb C\) of \(c\partial_z f + \bar c \partial_{\bar z} f = 0\). 

We have that
\begin{align}
\partial_z f & = z^2, \\
\partial_{\bar z} f & = -\frac{1}{\bar z^2}.
\end{align}

Hence, we have that \(\frac{c^2}{|c|^2} |z|^4 = 1\). Note that we only need to show that such an equation has 
no solutions with \(|c| = 1\) and \(z \in D\). So we consider \(c^2 |z|^4 = 1\). Note that every term other than
\(c\) is in \(\mathbb R\), and \(|z|^4 \geq 0\). Therefore we must have that \(c = \pm 1\), 
and therefore \(|z|^4 = 1\). Therefore there are no solutions in \(D\), and \(Df\) must be rank two 
throughout \(D\). Hence, \(f\) is a local diffeomorphism.

We will show that \(f(C)\) is a Jordan curve, and we will use a topological argument to show that \(f\)
is in fact a global diffeomorphism of \(D\) onto the exterior of \(f(C)\). To do so, we will employ
a winding number argument.

\subsubsection*{The Problem}
\begin{enumerate}
\item Show that \(f\) is a one-to-one map of the boundary component \(C = \{|z| = 1\}\).
\item For any \(w_0\) in the exterior of the Jordan curve \(f(C)\), use a winding number argument to 
show that \(w_0\) is the image of exactly one point in \(D\). 
\item Similarly show that no points \(w_0\) in the interior of \(f(C)\) are in the image of \(D\).  
\item Finally, argue that every point \(w_0 \in f(C)\) is not in the image \(f(D)\).
\end{enumerate}
Therefore we have that \(f\) is a global diffeomorphism of \(D\) onto the exterior of \(f(C)\). 

\subsubsection*{The Solution}
\begin{enumerate}
\item First let us show that \(f\) is a one-to-one map when restricted to \(C = \{|z| = 1\}\). On \(C\),
we have that \(\bar z = 1 / z\), so we have that \(f(z) = z + z^3 / 3\). 

So if \(|z| = |\zeta| = 1\) and \(f(z) = f(\zeta)\), then 
\begin{equation}
z - \zeta + \frac{z^3 - \zeta^3}{3} = 0. 
\end{equation}
We factor this equation by \(z - \zeta\) to get either \(z =\zeta\) or
\begin{equation}
3 + z^2 + z\zeta + \zeta^2 = 0.
\end{equation}
Now, \(|z^2| = |z\zeta| = |\zeta^2| = 1\); so \(|z^2 + z\zeta + \zeta^2| \leq 3\) with equality only when
all three terms are the same. So from the above we must have that all three terms are the same, and hence
\(z = \zeta\).

In either case \(z = \zeta\), and so we have that \(f\) is one-to-one when restricted to \(C\).

\item If \(w_0\) is in the exterior of the Jordan curve \(f(C)\), then notice that \(f(C)\) is a simply connected
curve in \(\mathbb C\setminus{w_0}\). 

However, for \(r\) small, the curve \(\theta \to f(re^{i\theta})\) is a curve with winding number \(1\) about
\(w_0\). 

Next, note that since \(f\) is a local diffeomorphism, the points in \(f^{-1}(w_0)\) must be separated. Since
\(|f| \to \infty\) as \(z \to 0\), we must then have that the image \(f^{-1}(w_0)\) consists of at
most finitely many \(k\) points, where \(k \geq 0\). 

We may homotope a counter-clockwise route \(\gamma\) on the circe \(C\) to a counter-clockwise circle around \(z = 0\),
counter-clockwise circles around each point of \(f^{-1}(w_0)\), and overlapping pairs of curves (running in 
opposite directions) that connect each of the former circles.

Since \(w_0\) is in the exterior of \(f(C)\), we have that the winding number of \(f(\gamma)\) is \(0\).
Furthermore, we know that \(f\) is orientation reversing near \(z = 0\) and so it must be orientation reversing on
\(D\) since it is a local diffeomorphism on \(D\). Therefore, we have that winding numbers of 
each of the above curves are \(-1\).

From the above homotopy, we then have that \(0 = k - 1\), and therefore we must have that \(k = 1\). So 
\(f^{-1}(w_0)\) consists of a single point, and so f is one to one onto the part of the image in the exterior
of \(f(C)\).

\item Similarly, when \(w_0\) in the interior of \(f(C)\), we may apply the above the argument. However, now the
winding number of \(f(\gamma)\) is \(1\). Therefore, we get \(1 = k - 1\), and so \(k = 0\). Hence, the image of
\(f\) does not intersect the interior of \(f(C)\). 

\item Finally note that there can't be any points \(z \in D\) such that \(f(z) \in f(C)\), because this would
imply that there are points in a neighborhood of \(f(C)\) that has more than one point in their pre-image. However,
this impossible by our work above. 
\end{enumerate}
