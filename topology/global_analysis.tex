\subsection{Global Analysis}
\subsubsection*{The Setup}

It is sometimes possible for us to use topology to extend an argument that holds in a limit to a more global case.

Here we look at an example from Osserman \cite{Osserman} where we can use winding number arguments to show that
a map is in fact a diffeomorphism. In particular, consider the domain \(D = \{0 < |z| < 1\} \subset \mathbb C\}\), 
and consider the map 
\begin{equation}
f(z) = \frac{1}{\bar z} + \frac{z^3}{3},
\end{equation} 
defined on \(D\). 

Let us next show that \(f\) is a local diffeomorphism. Recall that we by the inverse function theorem, we need to 
only show that on all points of \(D\) we have that \(Df\) is of rank two. This is equivalent to only having
trivial solutions \(r_i \in \mathbb R\) to \(r_1\partial_x f + r_2\partial_y f\); this is equivalent to only
a trivial solution \(c\in \mathbb C\) of \(c\partial_z f + \bar c \partial_{\bar z} f = 0\). 

We have that
\begin{align}
\partial_z f & = z^2, \\
\partial_{\bar z} f & = -\frac{1}{\bar z^2}.
\end{align}

Hence, we have that \(\frac{c^2}{|c|^2} |z|^4 = 1\). Note that we only need to show that such an equation has 
no solutions with \(|c| = 1\) and \(z \in D\). So we consider \(c^2 |z|^4 = 1\). Note that every term other than
\(c\) is in \(\mathbb R\), and \(|z|^4 \geq 0\). Therefore we must have that \(c = \pm 1\), 
and therefore \(|z|^4 = 1\). Therefore there are no solutions in \(D\), and \(Df\) must be rank two 
throughout \(D\). Hence, \(f\) is a local diffeomorphism.

We will use a topological argument to show that \(f\)
is in fact a global diffeomorphism of \(D\) onto its image. To do so, we will employ
a winding number argument.

\subsubsection*{The Problem}
\begin{enumerate}
\item Show that \(f\) is a one-to-one map of the boundary component \(C = \{|z| = 1\}\).
\item Show that image of \(f(\left\{|z| < 1\right\})\) is contained in the exterior of the Jordan curve
\(f(\{|z| = 1\})\).
\item For any \(w_0 \not \in f(C)\), use a winding number argument to show that there is exactly one point
\(z_0 \in D\) such that \(f(z_0) = w\).
\item Show that the above also holds for any \(w_0 \in f(C)\).
\end{enumerate}
Therefore we have that \(f\) is a global diffeomorphism of \(D\) onto its image \(f(D)\).

\subsubsection*{The Solution}
\begin{enumerate}
\item First let us show that \(f\) is a one-to-one map when restricted to \(C = \{|z| = 1\}\). On \(C\),
we have that \(\bar z = 1 / z\), so we have that \(f(z) = z + z^3 / 3\). 

So if \(|z| = |\zeta| = 1\) and \(f(z) = f(\zeta)\), then 
\begin{equation}
z - \zeta + \frac{z^3 - \zeta^3}{3} = 0. 
\end{equation}
We factor this equation by \(z - \zeta\) to get either \(z =\zeta\) or
\begin{equation}
3 + z^2 + z\zeta + \zeta^2 = 0.
\end{equation}
Now, \(|z^2| = |z\zeta| = |\zeta^2| = 1\); so \(|z^2 + z\zeta + \zeta^2| \leq 3\) with equality only when
all three terms are the same. So from the above we must have that all three terms are the same, and hence
\(z = \zeta\).

In either case \(z = \zeta\), and so we have that \(f\) is one-to-one when restricted to \(C\).

\item We first show that \(0 \not \in f(D)\). Note that when
\begin{equation}
0 = \frac{1}{\bar z} + \frac{z^3}{3},
\end{equation}
we have that \(-3 = z^2 |z|^2\). Therefore, at such a point, \(|z| > 1\), and so \(z \not \in D|\).

Now we claim that \(0\) is in the interior of \(f(\{|z| = 1\})\). If not, since \(|f(z)| \to \infty\)
as \(z \to 0\), we have that if \(0\) is in the exterior, then \(|f(z)|\) must obtain a minimum on \(D\)
(and not on \(\partial D\)). However, this is impossible since \(f\) is a local diffeomorphism on \(D\). 
Therefore \(0\) is in the interior of \(f(C)\).

Now, we show that \(f(D)\) is contained in the exterior to the Jordan curve \(f(C)\). We note that if this
wasn't the case, then connect such an interior point \(f(z_0)\) to \(0\) by a curve on the interior. Then
along this curve, there must be a point where the derivative \(Df\) has rank less than 2. Again, this
contradicts that \(f\) is a local diffeomorphism. Therefore, \(f(D)\) is contained in the exterior of \(f(C)\).

\item If \(w_0 \not \in f(C)\), then notice that \(f(C)\) is a simply connected
curve in \(\mathbb C\setminus{w_0}\). 

However, for \(r\) small, the curve \(\theta \to f(re^{i\theta})\) is a curve with winding number \(1\) about
\(w_0\). 

Next, note that since \(f\) is a local diffeomorphism, the points in \(f^{-1}(w_0)\) must be separated. 
\end{enumerate}
