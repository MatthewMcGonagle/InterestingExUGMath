\subsection{Example of Weak Convergence that is not Metrizable}

In this example we explore a normed vector space where weak convergence of its continuous linear functionals
is not given by a metric. 

\subsubsection*{The Setup}

Let \(V\) be the space of \(\mathbb R\) valued sequences \(v_i\) such that \(v_i \neq 0\) for at most a finite
number of \(i\); we give \(V\) the \(l^\infty\) norm where \(\|v\|_{l^\infty} = \sup_i |v_i|\). For example,
\( (1, 1, 0, 0, 0, ...) \in V\), but \((1, 1, 1, 1, ...)\not\in V\). 

Let \(V^*\) be the space of continuous linear functionals on \(V\) with respect to the \(l^\infty\) norm; i.e.
the linear functions \(\omega : V \to \mathbb R\) that satisfy the usual continuity condition
\(|\omega(v)| \leq C_\omega \|v\|_{l^\infty}\). 

A sequence \(\omega_n \in V^*\) is said to converge weakly to \(\psi \in V^*\) when for any fixed vector
\(v \in V\) we have that \(\omega_n(v) \to \psi(v)\). 

We wish to show that there is no metric \(d\) on \(V^*\) such that weak convergence on \(V^*\) is equivalent to
convergence with respect to the metric \(d\); therefore, weak convergence properly belongs to the notion
of convergence with respect to topology and not metric spaces. Note that weak convergence is covered by the
topology generated by sets \(U_{\psi, v, \delta}\) of the form
\(U_{\psi, v, \delta} \coloneq \{\omega \in V^* \mid |\omega(v) - \psi(v)| < \delta\}\). That is, it is
covered by the topology
generated by a family of semi-norms. 

To show our desired conclusion, let \(d\) be any metric on \(V^*\) such that if 
\(d(\omega_n, \psi) \to 0\) then \(\omega_n\) weakly converges to \(\psi\). We will show that there must
exist a sequence \(\omega_n\) weakly converging to \(0\), but \(\omega_n\) does not converge to \(0\) with
respect to the metric \(d\). 

\subsubsection*{The Problem}

Given a metric \(d\) on \(V^*\) such that if \(d(\omega_n, \psi) \to 0\) then \(\omega_n\) weakly converges
to \(\psi\). Show that there must exist a sequence \(\omega_n\) weakly converging to \(0\), but \(\omega_n\)
does not converge to \(0\) with respect to \(d\). 

\subsubsection*{The Solution}

For any index \(i \in \mathcal N\), let \(1_i \coloneq v \in V\) such that \(v_i = 1\) and \(v_j = 0\) for
\(j \neq i\).

Consider any finite subset \(S \subset \mathcal N\). We let \(\delta(S)\) be the infimum of all
\(d(\omega, 0)\) such that \(\omega(1_i) \geq 1\) for some \(i \in S\).

We claim that \(\delta(S) > 0\). Since \(S\) is finite, there exists a fixed \(i \in S\) and a sequence \(\omega_n\)
such that \(d(\omega_n, 0) \to \delta(S)\) and \(\omega_n(1_i) \geq 1\). Hence we have that \(\omega_n\) does
not weakly converge to \(0\). Therefore by our assumption on \(d\), we must have that \(\omega_n\) also does
not converge to \(0\) with respect to the metric \(d\). Hence, \(\delta(S) > 0\). 

Now partition \((0, \infty)\) into a countably infinite set of disjoint intervals,
each of which is bounded away from
zero: \([1, \infty)\), \([1/2, 1)\), \([1/4, 1/2)\), and so on. Since the set of all finite subsets
\(S \subset \mathcal N\) is uncountable infinite, we must have that one of the above intervals contains
\(\delta(S)\) for an infinite collection of \(S\). Therefore, we may find an infinite collection \(\mathcal C\)
of \(S\)
such that \(\delta(S) \geq \delta > 0\) for some fixed \(\delta > 0\) independent of \(S\) in \(\mathcal C\).

Since \(\mathcal C\) is infinite, we can find a strictly increasing sequence of integers \(n_i\) such that
\(n_i \in S\) for some \(S \in \mathcal C\). Then consider the sequence of continuous linear
functionals \(\omega_i(v) = v_{n_i}\). Note that \(\omega(1_{n_i}) = 1\). Therefore we have that
\(d(\omega_i, 0) \geq \delta\); hence \(\omega_i\) does not converge to \(0\) with respect to the metric \(d\).

However, for any fixed \(v \in V\), \(v_i\) is non-zero for only finitely many \(i\). Therefore, for
large enough \(i\) we have that \(v_{n_i} = 0\). Hence \(\omega_i(v) \to 0\). So \(\omega_i\) converges
weakly to \(0\) despite the fact that it does not converge to \(0\) with respect to \(d\). 
