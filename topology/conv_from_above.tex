\subsection{No Metric For Convergence From Above}
\subsubsection*{The Setup}

Here we look at another notion of convergence for which there is no metric \(d(x, y)\) that describes it.
In particular we will look at a notion of convergence for sequences in \(\mathbb R\) such that there is
no metric \(d(x, y)\) on \(\mathbb R\) 
where the convergence can be described using the metric space convergence of \(d\).

First let us define that a sequence \(a_n \in \mathbb R\) \textit{converges from above} to \(L\)\} when 
for every \(\epsilon > 0\), there exists an \(N\) such that if \(n\geq N\) then \(L \leq a_n < L + \epsilon\).

Roughly, the above definition means that in addition to converging to \(L\) (in the normal sense), 
eventually the sequence must also be above \(L\). Note that we do not assume that \(a_n\) satisfies any sort
of monotonicity (or eventual monotonicity). 

The fact that this is notion of convergence depends on a direction depends on a direction (i.e. from above)
gives one the intuition that this convergence must depend on more than distance. That is, it is no surprise
that a metric \(d(x, y)\) will not be enough to describe it.

However, there must be some care here. Note that there are subsets \(U \subset \mathbb R\) 
such that \textit{converges from above} for \(a_n \in U\)
can be described using the standard metric on \(\mathbb R\). For example, when 
\(U = \{0\} \cup \{1 / m | m \in \mathbb Z^+\}\), a sequence \(a_n \in U\) converges with respect to the
standard sub-space metric if and only if it converges from above. This can easily be seen, because any
sequence \(a_n \in U\) converging to some \(1 / m \in U\) must eventually be constant. Finally, if \(a_n\)
converges to \(0\), then by the defintion of \(U\) every point \(a_n\) already satisfies \(a_n \geq 0\).

From our above example of a sub-set \(U\), we see that non-existence of such a metric for \(\mathbb R\)
will be a little more deep than just the directionality. We will need to use the uncountable nature of
\(\mathbb R\).

\subsubsection*{The Problem}

Prove that there does not exists a metric \(d(x, y)\) on \(\mathbb R\) such that a sequence \(a_n \in \mathbb R\)
\textit{converges from above} to \(L\) if and only if the sequence converges to \(L\) in the sense of metric-space
convergence for \(d\).

\subsubsection*{The Solution}

We prove by contradiction. Assume first that such a metric \(d\) exists.

We claim that every point \(x \in \mathbb R\) has a \(\delta_x > 0\) such that all points \(y\) in the metric ball
\(B_{\delta_x}(x)\) satisfy \(x \leq y\). If this was not the case, then we would have a sequence of points \(a_n\)
converging to \(x\) with respect to the metric convergence of \(d\) but satisfying \(a_n < x\). This is a sequence
that can not converge to \(x\) from above; by our assumption on \(d\) this is impossible. 

We will now find a sequence that converges from above but does not converge with respect to the metric. Since
the open interval \((0, 1)\) is uncountably infinite, we
may find a strictly decreasing sub-sequence of points \(0 < a_n < 1\) such that \(a_n\)
converges to \(0 \leq x_0 \leq 1\)
and \(\delta_{a_n} > 1 / K\) for some \(K \in \mathbb Z^+\). Note that \(a_n\) converges from above to \(x_0\). 
So \(d(x_n, x_0) \to 0\). However \(a_n < a_m\) for \(m < n\), so \(a_n \not\in B_{1/K}(a_m)\).  

Since \(d(x_n, x_0) \to 0\), there exists \(a_m, a_n \in B_{1/3K}(x_0)\) and \(m < n\). However, we then have that
\(d(x_m, x_n) < 2 / 3K < 1/K\) which contradicts that \(a_n \not\in B_{1/K}(a_m)\).

