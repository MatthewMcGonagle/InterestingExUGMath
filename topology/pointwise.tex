\subsection{No Metric For Pointwise Convergence}

In this section we look at a motivating example for why metric spaces aren't enough to capture all notions
of convergence; a more abstract concept such as point-set topology is necessary to cover a simple example of convergence.

We will in particular consider pointwise convergence of a sequence of functions \(f_n(x) : [0, 1] \to \mathbb R\).
We will show that there is no metric \(d\) on these functions such that pointwise convergence of \(f_n(x)\) is
equivalent to \(d(f_n, f) \to 0\). In particular, we will show that given a metric \(d\) on these functions, there
exists a sequence of functions \(f_n\) pointwise converges to the zero function, but \(d(f_n, 0) \not \to 0\).

\subsubsection*{The Setup}

Historically, topology arose from a desire to have a universal framework to discuss different notions of convergence.
Since Weierstrass, analysts recognized that there are different notions of convergence for functions, e.g. uniform
convergence vs point-wise convergence. It was desirable to put all of these ideas on even footing.

In a sense, Frechet began the introduction of more abstract spaces with the creation of his so-called L-spaces in 1904. 
Later, in his doctoral dissertation in 1906, Frechet introduced the concept of a metric space. Convergence of
sequences was defined by abstract definitions of distance.

In 1914, Hausdorff took a more set-theoretic approach; he introduced the precursor to the modern topology. The abstract
concept of distance to a point is replaced with an abstract idea of neighborhoods of a point. For a nice historical
discussion of the creation of point-set topology, see \cite{mooreTopology}.

Now, let \(X\) be the set of \(\mathbb R\)-valued functions defined on \([0,1] \subset \mathbb R\). We will show that
there is no metrix \(d\) on \(X\) such that point-wise convergence is equivalent to convergence in the metric. Now,
at first you may think the issue is that the functions in \(X\) are unbounded, but this is not the issue. 

For example, the metric \(d_0(f, g) = \sup\limits_{x\in [0,1]} \min \{|f(x) - g(x)|, 1\}\) is a bounded metric on \(X\).
Furthermore, if \(d_0(f_n, g) \to 0\) then it isn't too hard to see that \(f_n\) must converg pointwise to \(g\) 
(in fact the convergence must be uniform). 

We will show that pointwise convergence for functions in \(X\) is not equivalent to convergence in any metric, but is
pointwise convergence equivalent to convergence in a topology? Yes, if we think of the space of functions \(X\) as a 
product of copies of \(\mathbb R\),  \(X = \prod\limits_{x\in [0,1]} \mathbb R_x\) where the \(x^\text{th}\) 
component of \(f\) is \(f(x)\), then the product topology will work. 

\subsubsection*{The Problem}

Let \(X\) be the \(\mathbb R\)-valued functions defined on \([0,1]\subset\mathbb R\). Show that there is no metric
such that a sequence of functions \(f_n \in X\) converges point-wise to \(g\) if and only if \(d(f_n, g) \to 0\). 
