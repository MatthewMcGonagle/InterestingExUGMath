\subsection{A Compact Subspace of Sequences of Non-negative Integers}

Here we consider a compact sub-space of a metric-space topology defined on the space of sequences of
non-negative integers. Both this metric space and sub-space are considered in \cite{federer} in relation to
Suslin sets. 

\subsection*{The Setup}

Let \(d_0(i, j)\) be the following bounded metric on the set of non-negative integers \(\mathbb Z_{\geq 0} = \{ 
0, 1, 2, ...\}\): 
\begin{equation}
d_0(i, j) \coloneqq \frac{|i -j|}{1 + |i - j|}.
\end{equation}
Note that \(d_0\) satisfies all of the conditions of being a metric, and that \(d(i, j) < 1\).

We let \(\mathcal N\) be the metric space of sequences of non-negative integers with following metric:
\begin{equation}
d(m, n) = \sum\limits_{i = 0}^\infty \frac{ d_0(m_i, n_i) } {2^i},
\end{equation}
for \(m, n \in \mathcal N\).

The importance of the space \(\mathcal N\) is that it is used to create the notion of Suslin sets which are
useful for studying the behavior of Borel sets under continuous maps \cite{federer}; these are topics
that are well beyond these notes and are not necessary for this example. We only mention them to provide
some context for \(\mathcal N\).

Next, given a fixed sequence \(M \in \mathcal N\), we will define the sub-space
\begin{equation}
K \coloneqq \{n \in \mathcal N \mid n_i \leq M_i \text{ for all } i\}.
\end{equation}
We will show that \(K\) is a compact sub-space of \(\mathcal N\).

Before we do so, let us discuss the topology of \(\mathcal N\). Note that \(d_0(i, j) \geq d_0(0, 1) = 1/2\)
when \(i \neq j\). Therefore, if \(d(m, n) < 2^{-j}\) for some \(j > 1\), then we know that \(m_i = n_i\)
for all \(i < j\). Therefore for every point \(m\) in the ball \(B_{2^{-j}}(n)\), we have that \(m_i = n_i\)
for all \(i  < j\).

You may be attempted to interpret members \(n \in \mathcal N\) as numeral expansions of numbers in \(\mathbb R\)
with an "infinite base". However, such an interpretation is not completely correct; let us look at such a
relationship and show that it isn't a homeomorphism.

Consider the interval \([0, 1) \subset \mathbb R\). We construct a function
\(g: \mathcal N \to [0, 1)\) that is onto and continuous. First partition the interval \([0, 1)\) into an
infinite set of intervals of length \(1/2, 1/4, 1/8, 1/16, ...\); explicitly the intervals are
\([0, 1/2), [1/2, 3/4), [3/4, 7/8), ...\). Then for \(n \in \mathcal N\), we let \(n_0\) choose one of the
preceding intervals. Next, we take our previously chosen interval, break it up into intervals of geometrically
decreasing sizes like we have done before, and let \(n_1\) choose the next interval. And we inductively
continue with all \(n_i\). This gives a continuous function \(g: \mathcal N \to [0, 1)\) that is onto.
Explicitly we have
\begin{equation}
g(n) = (1 - 2^{-n_0}) + 2^{-1 - n_0}(1 - 2^{-n_1}) + 2^{-2 - n_0 - n_1}(1 - 2^{-n_2}) + ... 
\end{equation} 
Note that \(g(n) < (1 - 2^{-n_0}) + 2^{-1-n_0} + 2^{-2-n_0} + ...\) and so we have that
\(g(n) < 1 - 2^{-n_0} + 2^{-n_0} = 1\). Similarly, by taking \(n_i \to \infty\), we see that
\(\sum_{n\in\mathcal N} g(n) = 1\). Finally, \(g(0) = 0\) and \(g(n) > 0\) for \(n \neq 0\).
Therefore, we have that
\(g: \mathcal N  \to [0, 1)\) is well defined.

Furthermore, \(g\) is continuous. If \(d(m, n) < 2^{-j}\), then \(m_i = n_i\) for \(i < j\). So we have that 
\begin{equation}
|g(m) - g(n)| < 2^{-j} + 2^{-j-1} + ... = 2^{1-j}.
\end{equation}
Therefore, we have that \(g\) is continuous.

From the desciption of the choice of intervals that are used to define \(g\), we see that if \(m < n\) for
\(m, n \in \mathcal N\), then \(g(m) < g(n)\); continuing the analogy to, e.g., a decimal expansion, this where
there is a difference. Two decimal expansions may represent the same number, e.g. \( 0.9999... = 1.0\).
However, since there is no upper limit on the "numerals" used to construct \(g : \mathcal N \to [0, 1)\), such
a situation is impossible for \(g\). Furthermore, the construction of \(g\) may be used to see that \(g\)
is onto \([0, 1)\).

We have shown that \(g: \mathcal N \to [0, 1)\) is a one-to-one, onto, and continuous. However, \(g^{-1}\) is
not continuous and so \(g\) fails to be a homeomorphism. This may be seen in the following way. Let
\(n^j = (0, j, 0, 0, ....) \in \mathcal N\) and \(m = (1, 0, 0, ...) \in \mathcal N\). Then we have that
\(g(n^j) \to g(m)\), but \(d(n^j, m) > 1/2\). Therefore \(g^{-1}\) is not continuous. Despite \(g\)
failing to completely characterize the topology of \(\mathcal N\), it can stil be useful to add to your 
understanding of the nature of \(\mathcal N\). 

