\subsection{A Compact Subspace of Sequences of Non-negative Integers}

Here we consider a compact sub-space of a metric-space topology defined on the space of sequences of
non-negative integers. Both this metric space and sub-space are considered in \cite{federer} in relation to
Suslin sets. 

\subsection*{The Setup}

Let \(d_0(i, j)\) be the following bounded metric on the set of non-negative integers \(\mathbb Z_{\geq 0} = \{ 
0, 1, 2, ...\}\): 
\begin{equation}
d_0(i, j) \coloneqq \frac{|i -j|}{1 + |i - j|}.
\end{equation}
Note that \(d_0\) satisfies all of the conditions of being a metric, and that \(d(i, j) < 1\).

We let \(\mathcal N\) be the metric space of sequences of non-negative integers with following metric:
\begin{equation}
d(m, n) = \sum\limits_{i = 0}^\infty \frac{ d_0(m_i, n_i) } {2^i},
\end{equation}
for \(m, n \in \mathcal N\).

The importance of the space \(\mathcal N\) is that it is used to create the notion of Suslin sets which are
useful for studying the behavior of Borel sets under continuous maps \cite{federer}; these are topics
that are well beyond these notes and are not necessary for this example. We only mention them to provide
some context for \(\mathcal N\).

Next, given a fixed sequence \(M \in \mathcal N\), we will define the sub-space
\begin{equation}
K \coloneqq \{n \in \mathcal N \mid n_i \leq M_i \text{ for all } i\}.
\end{equation}
We will show that \(K\) is a compact sub-space of \(\mathcal N\).

Before we do so, let us discuss the topology of \(\mathcal N\). Note that \(d_0(i, j) \geq d_0(0, 1) = 1/2\)
when \(i \neq j\). Therefore, if \(d(m, n) < 2^{-j}\) for some \(j > 1\), then we know that \(m_i = n_i\)
for all \(i < j\). Therefore for every point \(m\) in the ball \(B_{2^{-j}}(n)\), we have that \(m_i = n_i\)
for all \(i  < j\).

You may be attempted to interpret members \(n \in \mathcal N\) as numeral expansions of numbers in \(\mathbb R\)
with an "infinite base". However, such an interpretation is not completely correct; let us look at such a
relationship and show that it isn't a homeomorphism.

Consider the interval \([0, 1) \subset \mathbb R\). We construct a function
\(g: \mathcal N \to [0, 1)\) that is onto and continuous. First partition the interval \([0, 1)\) into an
infinite set of intervals of length \(1/2, 1/4, 1/8, 1/16, ...\); explicitly the intervals are
\([0, 1/2), [1/2, 3/4), [3/4, 7/8), ...\). Then for \(n \in \mathcal N\), we let \(n_0\) choose one of the
preceding intervals. Next, we take our previously chosen interval, break it up into intervals of geometrically
decreasing sizes like we have done before, and let \(n_1\) choose the next interval. And we inductively
continue with all \(n_i\). This gives a continuous function \(g: \mathcal N \to [0, 1)\) that is onto.
Explicitly we have
\begin{equation}
g(n) = (1 - 2^{-n_0}) + 2^{-1 - n_0}(1 - 2^{-n_1}) + 2^{-2 - n_0 - n_1}(1 - 2^{-n_2}) + ... 
\end{equation} 
Note that \(g(n) < (1 - 2^{-n_0}) + 2^{-1-n_0} + 2^{-2-n_0} + ...\) and so we have that
\(g(n) < 1 - 2^{-n_0} + 2^{-n_0} = 1\). Similarly, by taking \(n_i \to \infty\), we see that
\(\sum_{n\in\mathcal N} g(n) = 1\). Finally, \(g(0) = 0\) and \(g(n) > 0\) for \(n \neq 0\).
Therefore, we have that
\(g: \mathcal N  \to [0, 1)\) is well defined.

Furthermore, \(g\) is continuous. If \(d(m, n) < 2^{-j}\), then \(m_i = n_i\) for \(i < j\). So we have that 
\begin{equation}
|g(m) - g(n)| < 2^{-j} + 2^{-j-1} + ... = 2^{1-j}.
\end{equation}
Therefore, we have that \(g\) is continuous.

From the desciption of the choice of intervals that are used to define \(g\), we see that if \(m < n\) for
\(m, n \in \mathcal N\), then \(g(m) < g(n)\); continuing the analogy to, e.g., a decimal expansion, this where
there is a difference. Two decimal expansions may represent the same number, e.g. \( 0.9999... = 1.0\).
However, since there is no upper limit on the "numerals" used to construct \(g : \mathcal N \to [0, 1)\), such
a situation is impossible for \(g\). Furthermore, the construction of \(g\) may be used to see that \(g\)
is onto \([0, 1)\).

We have shown that \(g: \mathcal N \to [0, 1)\) is a one-to-one, onto, and continuous. However, \(g^{-1}\) is
not continuous and so \(g\) fails to be a homeomorphism. This may be seen in the following way. Let
\(n^j = (0, j, 0, 0, ....) \in \mathcal N\) and \(m = (1, 0, 0, ...) \in \mathcal N\). Then we have that
\(g(n^j) \to g(m)\), but \(d(n^j, m) > 1/2\). Therefore \(g^{-1}\) is not continuous. Despite \(g\)
failing to completely characterize the topology of \(\mathcal N\), it can stil be useful to add to your 
understanding of the nature of \(\mathcal N\). 

\subsection*{The Problem}
Given a fixed sequence \(m \in \mathcal N\), we define the sub-space
\(K \coloneqq \{n\in\mathcal N \mid n_i \leq m_i \text{ for all } i\}\). Show that \(K\) is compact.

\subsection*{The Solution}

It is very clear if \(\tilde K\) is the space corresponding to another fixed sequence \(\tilde m_i\) such that
\(m_i \leq \tilde m_i\) for all \(i\), then \(K\) is a closed subspace of \(\tilde K\). Since a
closed subspace of a compact
space must also be compact, without any loss in generality we may consider the case that \(m_i \geq 1\)
for all \(i\).

The key is to again interpret \(K\) as a numeral expansions as we did for \(g\) and \(\mathcal N\); however
this time, instead of having an "infinite base", we have a varying finite base given by the \(m_i\) (the base
will actually be \(1 + m_i\)). We will
use this to construct a function \(f: K \to [0, 1]\) that is continuous and onto. Note that the function
\(f\) will not be one to one, but the inverse image of any point will be finite; also note that it is
onto a closed interval. With \(f\) constucted, we can then imitate the proof
that a closed and bounded interval in \(\mathbb R\) is compact by looking at the set 
\(S \coloneqq \{x\in [0,1] \mid f^{-1}([0, x]) \text{ is compact }\}\). 

Interpreting \(K\) as numeral expansions for numbers in \([0, 1]\) with a variable base \(m_i\), we are lead
to the following construction for \(f(n)\) for \(n \in \mathcal N\). First, we divide \([0, 1]\)
into \(m_0 + 1\) intervals and choose the \(n_0^{\text{th}}\) interval. Next we divide that interval into
\(m_1  + 1\) intervals and choose the \(n_1^{\text{th}}\) sub-interval. We continue this for every \(n_i\).
We are lead to the following formula:
\begin{equation}
f(n) \coloneqq \frac{n_0}{1 + m_0} + \frac{n_1}{(1+m_0)(1+m_1)} + ...
\end{equation}
Note that
\begin{align}
f(n) & \leq \frac{m_0}{1 + m_0} + \frac{m_1}{(1 + m_0)(1 + m_1)} + ... \\
    & = 1 - \frac{1}{1+m_0} + \frac{m_1}{(1 + m_0)(1 + m_1)} + ... \\
    & = 1 - \frac{1}{(1+m_0)(1+m_1)} + ...
\end{align}
and the pattern continues. So we see that the sum of the first \(j+1\) terms is equal to
\(1 - \frac{1}{(1 + m_0)...(1+m_j)}\). Therefore, \(f(n)\) is well-defined by a convergent series and
\(f(n) \leq 1\). Furthermore, in the case that \(n_i = m_i\) for all \(i\), we have equality; so
\(f(m) = 1\). 

From the interval construction of \(f\), it is clear that \(f: K \to [0, 1]\) is also onto. Next note that
when \(d(x, y) < 2^{-j}\) for \(x,y \in K\), then \(x_i = y_i\) for \(i < j\). Hence
\begin{equation}
|f(x) - f(y)| \leq \frac{|x_j - y_j|}{(1 + m_0)...(1 + m_j)} + ...
\end{equation} 
Since \(|x_j - y_j| \leq m_j\), the analysis we used above will apply here as well. So we have that
\(|f(x) - f(y)| \leq \frac{1}{(1 + m_0)...(1 + m_j)} \leq 2^{-j-1}\), since \(m_i \geq 1\). Therefore,
\(f\) is continuous.

Finally, note that although \(f\) is not one to one, we do have the fact that \(f^{-1}(x)\) consists of at most
two points in \(K\) for any \(x \in [0, 1]\). Furthermore, if \(f^{-1}(x)\) consists of two points, then
they both must be of the form
\((n_0, ..., n_j, m_{j+1}, m_{j+2}, ...)\) and \((n_0, ..., n_j + 1, 0, 0, ...)\);
you may find it helpful to compare to how the decimal expansions \(0.999...\) and \(1.000...\)
represent the same real number.

With the above facts we have what we need to prove that \(K\) is compact. Consider the set
\(S \coloneqq \{x \in [0, 1] \mid f^{-1}(x) \text{ is compact }\}\), and let \(\mu = \sup S\). Note that
\(f^{-1}(0)\) consists of the single point \((0,0,...)\in K\); therefore \(0 \in S\). 

Next, we show that \(\mu \in S\).
