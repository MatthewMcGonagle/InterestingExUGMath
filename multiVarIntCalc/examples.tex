\section{Multivariable Integral Calculus}

\subsection{Function Not Satisfying Fubini's Theorem}

\subsubsection*{Set Up}

Here we consider Fubini's theorem in two-dimensions.

Fubini's theorem tells us two things:
\begin{itemize}
\item When we know we can compute a two-dimensional integral as a repeated application of one-dimensional integration over the variables \(x\) and \(y\).

\item When we know the order of the repeated one-dimensional integration over \(x\) and \(y\) doesn't depend on the order of integrating over \(x\) and \(y\). 

\end{itemize}

We will consider the problem of finding a simple function that doesn't satisfy Fubini's theorem. In particular, the result of applying the repeated 
one-dimensional integration will depend on the order of \(x\) and \(y\). We will aim to find a simple elementary function, and we will keep the domain of integration
simple, i.e. the square \(S = \{0 \leq x \leq 1 \text{ and } 0 \leq y \leq 1\}\).

As a hint as to how this process will work, let us recall the fact that for a convergent series \(\sum_i a_i\), the limit of the partial sums is independent of the order of
the sum when the series is absolutely convergent, i.e. \(\sum_i |a_i| < \infty\). For our problem of finding an appropriate function \(u(x,y)\), we are then lead to consider
finding a function \(u(x,y)\) that satisfies the following:
\begin{itemize}
\item The function \(u(x,y)\) takes positive and negative values. 
\item The integral of the aboslute value of \(u\) is not convergent, i.e. \(\iint_s |u| \dA = \infty\).
\item The integrals of the positive and negative parts of the function must cancel out in some way such that the repeated integration gives nice finite values despite
the fact that \(\iint_s |u| \dA = \infty\).
\end{itemize}


\subsubsection*{The Problem}

Find a nice elementary function \(u(x, y)\) defined on the square \(s = \{0 \leq x \leq 1 \text{ and } 0 \leq y \leq 1\}\) such that \(u(x,y)\) doesn't satisfy Fubini's theorem in the following sense:
\begin{itemize}
\item Both of the repeated integrals
\begin{equation}
\int\limits_0^1 \int\limits_0^1 u(x,y) \dx \dy,
\end{equation}
and
\begin{equation}
\int\limits_0^1 \int\limits_0^1 u(x,y) \dy \dx,
\end{equation}
exist and are finite.
\item However, the repeated integrals mentioned above are NOT equal.
\end{itemize}

\subsubsection*{Solution}

To keep things simple, we find a function \(u(x,y)\) that blows up to both \(\pm \infty\) at the corner \((0,0)\). First let us observe that the function 
\begin{equation}
f(x,y) = \frac{-1}{(x+y)^2}
\end{equation}
has \(\iint_S |f| \dA = \infty\); you can quickly see that convergence of this integral is suspect because \(|f|\) of order \(r^{-2}\) as the radius \(r \to 0\). This is the edge case of convergence
for two-dimensions (recall that the area element for polar coordinates in two-dimensions includes an extra \(r\), i.e. \(\dA = r d\theta dr\)). 

However, \(f(x,y)\) is always negative inside the square \(S\); so we won't get the cancellation of positive and negative parts that we desire. To fix this we set up \(u(x,y)\) to be a difference
of \(f(x,y)\) and a similar function. First, let \(A, B\) be constants that we will determine later. Then we use
\begin{equation}
u(x,y) = \frac{1}{(Ax + By)^2} - \frac{1}{(x+y)^2} .
\end{equation}

We need that \(u\) takes positive and negative values in \(S\). To make sure \(u\) is always defined in \(S\) we will restrict to considering \(A, B > 0\).

Next, consider the values of \(u\) along the line \(\{x + y = 1\}\). This line joins two corners of \(S\), i.e. \((0,1)\) and \((1,0)\). We will design \(A\) and \(B\) to make
sure \(u\) has opposite signs at these two corners. To do so, we need to compare the sizes of \(Ax + By\) and \(x + y\) at these two corners.

To get opposite signs, we need that \(Ax + By\) is above \(1\) at one of these two corners and below \(1\) at the other corner. At \((0,1)\), \(Ax + By = B\) and at \((1,0)\), \(Ax + By = A\).
So we can choose \(A\) is above \(1\) and \(B\) is below \(1\). A convenient choice is \(A = 2\) and \(B = 1/2\) (if you dive deeper into the construction, you will find that the reciprocal nature of \(A\) and \(B\) is also necessary, but we won't go into detail on this). 

So we have that
\begin{equation}
u(x,y) = \frac{1}{(2x + y/2)^2} - \frac{1}{(x+y)^2}.
\end{equation}
Let us verify that this function \(u(x,y)\) satisfies the conditions on the repeated integrals that we are looking for.

First, note that for any \(y > 0\), we have
\begin{align}
\int\limits_0^1 \frac{1}{(2x + y/2)^2} - \frac{1}{(x+y)^2} \dx & = \left.\frac{1}{x+y} - \frac{1}{2(2x + y/2)}\right|^1_{x = 0}, \\ 
    & = \frac{1}{1 + y} - \frac{1}{y} - \frac{1}{4+y} + \frac{1}{y}, \\
    & = \frac{1}{1+y} - \frac{1}{4+y}.
\end{align}
So we get that
\begin{align}
\int\limits_0^1 \left(\int\limits_0^1 \frac{1}{(2x + y/2)^2} - \frac{1}{(x+y)^2} \dx\right) \dy & = \int\limits_0^1 \frac{1}{1+y} - \frac{1}{4 + y} \dy, \\
    & = \left. \log(1 +y) - \log(4 + y)\right|_{y = 0}^1, \\
    & = \log(2) - \log(1) - \log(5) + \log(4), \\
    & = 3\log(2) - \log(5). 
\end{align}

Now, let us consider the other iterated integral. First, for any \(x > 0\), we have that
\begin{align}
\int\limits_0^1 \frac{1}{(2x + y/2)^2} - \frac{1}{(x+y)^2} \dy & = \left.\frac{1}{x+y} - \frac{2}{2x + y/2}\right|_{y = 0}^1, \\
    & = \frac{1}{x+1} - \frac{1}{x} -\frac{2}{2x+1/2} + \frac{2}{2x}, \\ 
    & = \frac{1}{x+1} - \frac{2}{2x+1/2}.
\end{align}
So we have that
\begin{align}
\int\limits_0^1 \left(\int\limits_0^1 \frac{1}{(2x + y/2)^2} - \frac{1}{(x+y)^2} \dy\right) \dx & = \int\limits_0^1 \frac{1}{x+1} - \frac{2}{2x+1/2} \dx, \\
    & = \left. \log(x+1) - \log(2x + 1/2)\right|_{x = 0}^1, \\
    & = \log(2) - \log(1) - \log(5/2) + \log(1/2), \\
    & = \log(2) - \log(5). 
\end{align}
And so we see the repeated integrals are finite, but do NOT match.
