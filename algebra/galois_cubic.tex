\subsection{Solving Cubic Equations Using Galois Theory}

\newcommand {\Gal}{\operatorname{Gal}}

\subsubsection*{The Setup}

In this example we explore using Galois Theory to explicitly solve cubic equations.
Note, plan to go beyond just showing that the cubic is solvable; we will actually
solve for the roots.

Consider a general cubic polynomial
\begin{equation}
p(x) = x^3 + Ax^2 + Bx + C.
\end{equation}
We are considering this as a polynomial over the field \(K = \mathbb Q(A, B, C)\), and 
\(A, B, C\) to be general non-algebraic elements. Let the roots of \(p\) be \(a, b, c\),
no particular relation to the capital versions that make up the coefficients of \(p\).

We will use without proof that \(\Gal K(a, b, c) / K = S_3\), the symmetric group of three
elements.

Recall that the classical method of solving cubic polynomials involves solving an auxilliary
quadratic equation. This quadratic equation could have roots that are complex even if the
roots of our particular cubic equation are real. This is infact the original historical motivation
for the construction of the imaginary number \(i = \sqrt{-1}\).

Galois' idea for approaching the solutions of polynomials is manifold:
\begin{itemize}
\item Our aim is to reduce Galois group to the trivial group by extending the base field. It is exactly
in this case that the roots must be contained in the extended base field. 

\item Adjoining the roots of an auxilliary equation does not by itself change the Galois group. Consider 
\(\tilde K = K(\alpha_1, \alpha_2, ..., \alpha_n)\) where the \(\alpha_i\) are
roots of an auxialliary equation with no factors in common with the original polynomial \(p(x)\). We expect
that \(\Gal \tilde K(a, b, c) / \tilde K = \Gal K(a, b, c) / K\). 

\item We adjoin roots of auxilliary equations, because they allow us to write down expressions that:
    \begin{itemize}
    \item Are expressed in terms of the roots.
    \item Are invariant under a normal sub-group of the current Galois group.
    \item Are not invariant under the complete Galois Group.
    \item The images of the quantity under the Galois group can be combined with primitive roots of
    unity to form a new quanitity such a simple power is invariant under the entire current Galois Group.
    This allows us
    to express the new quantity as a root of terms involving only the current base field \(\tilde K\).

    This
    will in turn allow us to express the new quantity in terms of other known quantities: coefficients
    of the original polynomial \(p(x)\) and roots of our auxialliary polynomials. Recall that we
    don't actually know the values of \(a, b, c\) yet, and so writing the expression just in terms
    of \(a, b, c\) doesn't give us a quantity that we actually know. 
    \end{itemize}

For example, consider we have an extension \(\tilde K\) of \(K\) such that 
\(\Gal \tilde K(a, b, c) / \Gal \tilde K\) is generated by the cyclic permutations of the roots
\(a, b, c\), and \(\tilde K\) contains the primitive third-roots of unity. Let \(\zeta_3\) be a
primitive third-root of unity. Consider the quanitity \(\beta = a\); this satisfies that: 
    \begin{itemize}
    \item \(\beta\) is NOT preserved by \(\Gal \tilde K(a, b, c) / \Gal \tilde K\); 
    \item \(\beta\) IS trivially preserved by the trivial normal subgroup that only consists
    of the identity.
    \item Note that the images of \(\beta\) under the Galois group are all of the roots \(\{a, b, c\}\).
    So we form the new quantity \(\alpha = a + b\zeta_3 + c\zeta_3^2\). The quantity \(\alpha\) is
    also NOT preserved by the Galois Group; however, since the Galois Group is generated by cyclic
    permutations, we see that the images of \(\alpha\) under the Galois Group are 
    \(\{\alpha, \alpha \zeta_3, \alpha \zeta_3^2\}\). 

    So, the power \(\alpha^3\) IS preserved by \(\Gal \tilde K(a, b, c) / \tilde K\). So we
    may write \(\alpha^3\) as an expression in \(\tilde K\). Later we will see that in practicality,
    this amounts to writing \(\alpha^3\) as a polynomial in quantities that are known, e.g. quantities
    like \(C = -abc\).
    \end{itemize} 
\end{itemize}

\subsubsection*{The Problem}

Use Galois theory to find the roots of the general cubic
\begin{equation}
p(x) = x^3 + Ax^2 + Bx + C,
\end{equation}
where the coefficients are in \(\mathbb Q\).

\subsubsection*{The Solution}

Recall that we treat the coefficients as independent non-algebraic elements. So our initial base field is
\(K = \mathbb Q(A, B, C)\). We use without proof that \(\Gal K(a, b, c) / K = S_3\), and is generated by
all of the permutations of \(a, b, c\).

The first normal sub-group of \(S_3\) is the alternating group \(A_3\), which also happens to be sub-group of
cyclic permutations of \(a, b, c\). To reduce the Galois group to \(A_3\) we seek to extend \(K\) by a quantity
\(\Delta\) such that \(\Delta\) is preserved by \(A_3\), but is NOT preserved by all of \(S_3\).

We make use of the classical quantity associated with the alternating group:
\begin{equation}
\Delta = (a - b)(a - c)(b - c).
\end{equation}
For any permutation \(\sigma\in\S_3\) we have that \(\sigma \Delta = \Delta\) exactly when \(\sigma \in A_3\) and
\(\sigma \Delta = - \Delta\) otherwise. So \(\Delta\) satisfies the properties we are interested in.

However, note that \(\Delta\) is expressed in terms of the roots \(a, b, c\), which are currently unknown. How
can we find a quantity that is actually known to us? Let \(\zeta_2\) be the primitive square-root of unity, i.e.
\(\zeta_2 = -1\). We can combine powers of \(\Delta\) with the images of \(\Delta\) under \(S_3\) to get
a value that is expressed as a root of a quantity that is invariant under \(A_3\). Note that the images of
\(\Delta\) are \(\{\Delta, -\Delta\}\). So we consider the quantity
\begin{align}
\alpha & = \Delta + (-\Delta)\zeta_2, \\
    & = \Delta + (-1)^2 \Delta, \\
    & = 2\Delta.
\end{align}

We have that \(\alpha^2\) is preserved by \(\Gal K(a, b, c) / \Gal K\), and similarly \(\Delta^2\) is also
preseved by the Galois Group. So we extend by the roots of the auxilliary polynomial \(q(x) = x^2 - \Delta^2\).
This amounts to just extending by \(\Delta\) itself.

So our first extension \(K(\Delta)\).

Now, we need to express \(\Delta^2\) in terms of our only known quantities, the coefficients of the polynomial
\(A, B, C\). First, we will need how \(A, B, C\) are related to the roots 
(despite the fact that we don't know what the roots are):
\begin{align}
A & = -(a + b + c), \\
B & = ab + ac + bc, \\
C & = -abc.
\end{align}
You may recognize these as being, within a change of sign, classical fundamental symmetric polynomials.
In fact, our method is related
to the classical fact a symmetric polynomial in \(a, b, c\) can be expressed as a polynomial in the classic
symmetric polynomials:
\begin{align}
\sigma_1(a, b, c) & = a + b + c, \\
\sigma_2(a, b, c) & = ab + ac + bc, \\
\sigma_3(a, b, c) & = abc.
\end{align}
So \(A = -\sigma_1, B = \sigma_2, C = -\sigma_3\).

Since \(\Delta^2 \in K\), we have that \(\Delta^2\) is a rational expression of \(A, B, C\). This implies
that \(\Delta^2\) is also a rational expression of \(\sigma_1, \sigma_2, \sigma_3\). However, \(\Delta^2\)
is a polynomial in \(a, b, c\), and \(\sigma_1, \sigma_2,\) and \(\sigma_3\) are also polynomials in
\(a, b, c\). The only way this is possible for general \(a, b, c\) is that \(\Delta^2\) is
actually a polynomial expression of \(\sigma_1, \sigma_2,\) and \(\sigma_3\).

Now, the fact that \(\Delta^2\) is a six-degree polynomial in \(a, b, c\) will put restrictions on the 
possible polynomial expressions of \(\sigma_1, \sigma_2, \sigma_3\). We have that
\begin{equation}
\Delta^2 = d_1 \sigma_3^2 + d_2 \sigma_3 \sigma_2 \sigma_1 + d_3 \sigma_3 \sigma_1^3
    + d_4 \sigma_2^3 + d_5 \sigma_2^2 \sigma_1^2 + d_6 \sigma_2 \sigma_1^4 + d_7 \sigma_1^6,
\end{equation}
for constants \(d_i \in \mathbb Q\).

Now, initially this seems like a huge mess and a complete headache to solve. However, we will see that by looking
at the right powers of the resulting equations, it isn't as bad as it initially looks.
Consider the largest power of \(c\) on the left hand side: \((a-b)^2 c^4\). Compare this to the right
hand side (each \(\sigma_i\) can only contribute at most one power of \(c\)).
The largest power of \(c\) for the right hand side is seen to be \(d_7 c^6\). So we have \(d_7 = 0\).

Now find the new largest power of \(c\) for the right hand side, \(d_6 (a + b)c^5\). Again, we must then
have that \(d_6 = 0\).

Repeat to get a largest power \(d_3 abc^4 + d_5 (a + b)^2 c^4\). Comparing to the left
hand side, we then have that \((a - b)^2 = d_3 ab + d_5 (a + b)^2\). Then we get \(d_5 = 1\)
and \(d_3 = -4\).
So we have
\begin{equation}
\Delta^2 = d_1 \sigma_3^2 + d_2 \sigma_3 \sigma_2 \sigma_1 
    - 4 \sigma_3 \sigma_1^3 + d_4 \sigma_2^3 + \sigma_2^2 \sigma_1^2.
\end{equation}

Next, look at the smallest powers of \(c\). The left hand side has lowest power \((a-b)^2a^2b^2\), note
that the term is independent of \(c\). The right hand side has \(d_4 a^3b^3 + a^2b^2(a + b)^2\); note that
terms like \(d_1 \sigma_3^2\) never contribute a term independent of \(c\). So we
must have that \((a-b)^2a^2b^2 = d_4 a^3b^3 + a^2b^2(a + b)^2\). Writing it out, we get
\begin{equation}
a^4b^2 - 2a^3b^3 + a^2b^4 = d_4 a^3b^3 + a^4b^2 + 2a^3b^3 + a^2b^4.
\end{equation}
So we see that \(-2 = d_4 + 2\) and so \(d_4 = -4\). Hence we have that
\begin{equation}
\Delta^2 = d_1 \sigma_3^2 + d_2 \sigma_3 \sigma_2 \sigma_1 
    - 4 \sigma_3 \sigma_1^3 - 4 \sigma_2^3 + \sigma_2^2 \sigma_1^2.
\end{equation}

Finally, let us plug in \(a = b = 1\) and \(c = -2\). So we have that \(\Delta = 0\) and \(\sigma_1 = 0\).
We get that
\begin{align}
0 & = 4d_1 - 4 (1 - 2 - 2)^3, \\
  & = 4d_1 + 4 (27) 
\end{align}
So we have that \(d_1 = -27\).
Therefore,
\begin{equation}
\Delta^2 = -27 \sigma_3^2 + d_2 \sigma_3 \sigma_2 \sigma_1 
- 4 \sigma_3 \sigma_1^3 - 4 \sigma_2^3 + \sigma_2^2 \sigma_1^2.
\end{equation}
Finally plug in \(a = b = c = 1\) to get that \(0 = -27 + 9d_2 - 4 (27) - 4 (27) + 3^4\). So
\(d_2 = 3 + 12 + 12 - 9 = 18\). So we finally get that
\begin{equation}
\Delta^2 = -27 \sigma_3^2 + 18 \sigma_3 \sigma_2 \sigma_1 
- 4 \sigma_3 \sigma_1^3 - 4 \sigma_2^3 + \sigma_2^2 \sigma_1^2.
\end{equation}
Then, rewriting in term of the original coefficients, we have that
\begin{equation}
\Delta^2 = -27 C^2 + 18 ABC
- 4 A^3C - 4 B^3 + A^2 B^2.
\end{equation}

Now, we know \(\Delta\) as a radical expression of the original equation and we have that
\(\Gal K(a, b, c) / K(\Delta) = A_3\), the cyclic permutations. The next normal sub-group
is simply the trivial group. An expression of the roots that is invariant under the trivial
group is simply one of the roots, say \(\beta = a\). This has images \(\{a, b, c\}\) under
the cyclic permutations. 

Now, let \(\zeta_3\) be a primitive third-root of unity. Then we have
that the quantity \(\alpha = a + b\zeta_3 + c\zeta_3^2\) satisfies that \(\alpha^3\) is
invariant under the cyclic permutations while \(\alpha\) itself is not. The problem is
that we can't use \(\alpha\) yet in our extension as \(\zeta_3 \not \in K(a, b, c)\).

So, we must extend \(K\) itself so that we are able to use \(\zeta_3\). So let
\(\tilde K = K(i\sqrt{3})\); note that \(\tilde K\) contains \(\zeta_3\). We then have that
\(\Gal \tilde K(a, b, c) / \tilde K(\Delta) = \Gal K(a, b, c) / K(\Delta)\), generated by
the same permutations of \(a, b, c\).

So, we have that \(\alpha^3 \in \tilde K(\Delta)\). However, we can save ourselves some work
by going a little further
in narrowing down which field it resides in. Consider the automorphism of 
\(\tau \in \Gal \tilde K(a, b, c) / \tilde K\) generated by 
\begin{equation}
\begin{cases}
    \tau a = a,\\
    \tau b = c, \\
    \tau c = b, \\
    \tau i\sqrt{3} = -i\sqrt{3}.
\end{cases}
\end{equation}
Note that \(\tau\) restricts to an automorphism of \(\tilde K(\Delta)\). Furthermore, we have
that \(\tau \alpha = \alpha\). Therefore, \(\alpha^3\) is in the fixed field of the group
generated by \(\tau\), i.e. \(\{\text{id}, \tau\}\). What is the fixed sub-field of \(\tilde K(\Delta)\)? 

The automorphism \(tau\) takes an element \(k_1 + k_2 i\sqrt{3} + k_3 \Delta + k_4 \Delta i\sqrt{3}\) to
\(k_1 - k_2 i\sqrt{3} - k_3 \Delta + k_4 \Delta i\sqrt{3}\). Therefore we see that the fixed sub-field
is \(K(\Delta i\sqrt{3})\). Hence \(\alpha^3 \in K(\Delta i\sqrt{3})\).

Since \(\alpha^3\) is a polynomial in 
\(a, b, c\) and \(i\sqrt{3}\), we may express it using polynomial functions of
\(\sigma_1, \sigma_2\), and \(\sigma_3\) instead of more general
rational expressions; furthermore, we need to only use the standard expansion for \(\Delta i\sqrt{3}\).
Now, also note that \(\alpha^3\) is a homogeneous polynomial in \(a, b, c\) of degree 3, and that 
\(\Delta = (a - b)(a - c)(b - c)\) is also homogeneous of degree 3. So we have that

\begin{equation}
\alpha^3 = e_1 \sigma_3 + e_2 \sigma_2 \sigma_1 + e_3 \sigma_1^3
    + e_4 \Delta i\sqrt{3},
\end{equation}
where the \(e_i \in \mathbb Q\).

We play a similar game as before to find the \(e_i\). Find the largest powers of \(a\) on the left and right
hand sides. For the left hand side, we have that the largest power of \(a\) is \(a^3\). For the right hand 
side, we get \(e_3 a^3\). Therefore, \(a^3 = e_3 a^3\), and so \(e_3 = 1\). 

Now look at the lowest powers of \(a\). The left hand side gives us 
\((b\zeta_3 + c\zeta_3^2)^3 = b^3 + 3\zeta_3b^2c + 3\zeta_3^2 bc^2 + c^3\). The right hand side gives us
\(e_2 bc(b + c) + (b + c)^3 + e_4 bc(b - c)i\sqrt{3}
    = b^3 + (e_2 + 3 + e_4i\sqrt{3})b^2c + (e_2 + 3 - e_4 i\sqrt{3})bc^2 + c^3\).

So we get that
\begin{equation}
\begin{cases}
    3\zeta_3 = e_2 + 3 + e_4i\sqrt{3}, \\
    3\zeta_3^2 = e_2 + 3 - e_4i\sqrt{3}.
\end{cases}
\end{equation}
So we get that \(2e_2 + 6 = 3(\zeta_3 + \zeta_3^2) = -3\). Therefore \(e_2 = -9/2\).

We also get that \(e_4 2i\sqrt{3} = 3(\zeta_3 - \zeta_3^2) = 3i\sqrt{3}\). So \(e_4 = 3 / 2\). 

So we have that
\begin{equation}
\alpha^3 = e_1 \sigma_3 - \frac{9}{2} \sigma_2 \sigma_1 + \sigma_1^3
    + \frac{3}{2} \Delta i\sqrt{3}.
\end{equation}

Now, plug in \(b = c = 1\) and \(a = -2\). We have that \(\sigma_1 = \Delta = 0\). We get
\begin{equation}
(-2 + \zeta_3 + \zeta_3^2)^3 = (-2 - 1)^3 = -27,
\end{equation}
and so
\begin{equation}
-27 = -2 e_1.
\end{equation}
So \(e_1 = 27/2\). So we finally have that
\begin{equation}
\alpha^3 = \frac{27}{2} \sigma_3 - \frac{9}{2} \sigma_2 \sigma_1 + \sigma_1^3
    + \frac{3}{2} \Delta i\sqrt{3}.
\end{equation}
Rewriting in terms of the original coefficients, we have that
\begin{equation}
\alpha^3 = -\frac{27}{2} C + \frac{9}{2} AB - A^3
    + \frac{3}{2} \Delta i\sqrt{3}.
\end{equation}

So we have that \(\tilde K(\Delta, \alpha)\) is a splitting field for our original polynomial \(p(x)\).
That is, \(\tilde K(a, b, c) = \tilde K(\Delta, \alpha)\). However, to minimize our work in identifying
the roots, we want to find \(K(a) \subset \tilde K(a, b, c)\). Note that 
\(\Gal \tilde K(a,b,c) / K(a)\) is of order four and generated by two automorphisms:
the first sending \(i\sqrt{3} \to -i\sqrt{3}\) and the second sending \(b \to c\). To find the
sub-field \(K(a)\) we need to describe the fixed field in terms of the original 
polynomial coefficients.

We note that the images of \(\alpha\) under this group of autormorphisms consists of \(\alpha\) itself
and the element \(\beta = a + b\zeta_3^2 + c \zeta_3\). It is immediately clear that 
\(\alpha + \beta = 2a - b -c\) is fixed by this group of automorphisms, but it is NOT fixed by the 
automorphisms generated by the cyclic permutations.
So we then have that \(K(a) = K(\alpha + \beta)\).

How can we compute \(\beta\)? Notice that the
automorphisms in \(\Gal \tilde K(\Delta, \alpha) / \tilde K(\Delta)\), i.e. those generated by the
cyclic permutations, preserve the product \(\alpha\beta\). Therefore, \(\alpha\beta \in \tilde K(\Delta)\). 
Furthermore, it is preserved by the automorphism generated by the transposition switching \(b\) and \(c\).
From the structure of \(S_3\), this is enough to give us that \(\alpha\beta\) is preserved by the entirity
of \(\Gal K(a,b,c) / K\). Therefore, \(\alpha\beta \in K\).

Since \(\alpha\beta\) is a homogenous polynomial in \(a, b, c\) of degree 2, we have that
\begin{equation}
\alpha\beta = f_1 \sigma_2 + f_2 \sigma_1^2, 
\end{equation}
where \(f_i \in \mathbb Q\).

For the highest power of \(a\), on the left hand side we get \(a^2\) and on the right hand side we
get \(f_2 a^2\). Therefore \(f_2 = 1\).

Now, look at the lowest power of \(a\). On the left hand side we get \(b^2 + (\zeta_3 + \zeta_3^2)bc + c^2\).
On the right hand side we get \(f_1 bc + (b + c)^2\). So we get that \(-1 = f_1 + 2\), and that \(f_1 = -3\).
Therefore,
\begin{equation}
\alpha\beta = -3 \sigma_2 + \sigma_1^2.
\end{equation}
In terms of the original coefficients,
\begin{equation}
\alpha\beta = -3B + A^2.
\end{equation}

Now, we wish to find one the root \(a \in K(a) = K(\alpha + \beta)\). Note that \(a\) is itself
a homogeneous linear polynomial of degree one in the roots, and
\(\alpha + \beta\) is homogeneous of degree 1. So we are left with the following possibilities:
\begin{equation}
a = g_1 \sigma_1 +g_2 (\alpha + \beta),
\end{equation}
where \(g_i \in \mathbb Q\).

Next, realize that \(\alpha + \beta = 2a - b - c\). So we see that
\begin{equation}
a = \frac{1}{3} \sigma_1 + \frac{1}{3} (\alpha + \beta).
\end{equation}
Rewriting in terms of the coefficients of the original polynomial we get
\begin{equation}
a = -\frac{1}{3} A + \frac{1}{3}(\alpha + \beta).
\end{equation}
To get the rest of the roots, simply apply the automorphism of \(K(a, b, c)\) generated by cyclic permutations.
We get that
\begin{equation}
\begin{cases}
a = -\frac{1}{3} A + \frac{1}{3}(\alpha + \beta), \\
b = -\frac{1}{3} A + \frac{1}{3}(\zeta_3^{-1} \alpha + \zeta_3 \beta), \\
c = -\frac{1}{3} A + \frac{1}{3}(\zeta_3 \alpha + \zeta_3^{-1} \beta). 
\end{cases}
\end{equation}

Now, in practice we can't be sure we have selected the precise \(\Delta\) and \(\alpha\) we have used above.
For example, we can only by sure that we select \(\hat\Delta \in \{-\Delta, \Delta\}\). Similarly, depending
on our choice of \(\hat \Delta\), we select \(\hat \alpha \in \{\alpha, \zeta_3\alpha, \zeta_3^2\alpha,
\beta, \zeta_3\beta, \zeta_3^2\beta\}\). However, everything will be okay as long as we enforce the
following:
\begin{align}
\hat\Delta^2 & = -27 C^2 + 18 ABC
    - 4 A^3C - 4 B^3 + A^2 B^2, \\
\hat\alpha^3 & = -\frac{27}{2} C + \frac{9}{2} AB - A^3
    + \frac{3}{2} \hat\Delta i\sqrt{3}, \\
\hat\alpha \hat\beta & = -3B + A^2, \\
a & = -\frac{1}{3} A + \frac{1}{3}(\hat\alpha + \hat\beta), \\
b & = -\frac{1}{3} A + \frac{1}{3}(\zeta_3^{-1} \hat\alpha + \zeta_3 \hat\beta), \\
c & = -\frac{1}{3} A + \frac{1}{3}(\zeta_3 \hat\alpha + \zeta_3^{-1} \hat\beta). 
\end{align}
