\subsection{Cardano's Method of Solving Cubic Equations}

In this section we will look at Cardano's method for solving cubic equations using a specific example. We
will motivate the algebraic manipulations using Cardano's geometric manipulations. For a reference on Cardano's
point of view, see \cite{cardano}. 

\subsubsection*{The Setup}

\newcommand{\boxDim}[3]{{#1} \times {#2} \times {#3}}
Cardano's method is based on geometry. Consider the equation
\begin{equation}
x^3 = 6x^2 + x + 2.
\end{equation}
Cardano considers this as a geometric problem of matching volumes. The left hand side represents
the volume of a cube of side lengths \(x\). Each term on the right hand side represents a specific volume.
So we have

\begin{tabular}{p{0.45\textwidth} | p{0.45\textwidth}}
    Left & Right \\
    \begin{itemize}
    \item Cube with side lengths \(x\).
    \end{itemize} 
    
    &
    \begin{itemize}
    \item Box of dimensions \(\boxDim{6}{x}{x}\). 
    \item Box of dimensions \(\boxDim{1}{1}{x}\).
    \item Cube of side lengths \(2^{1/3}\). 
    \end{itemize} 
\end{tabular}

Now, Cardano's idea is to break up the length \(x\) into two lengths \(a\) and \(b\). That is we, look at
\(x = a + b\).

When we do so, we get

\begin{tabular}{p{0.45\textwidth} | p{0.45\textwidth}}
    Left & Right \\
    \begin{itemize}
    \item Cube of side lengths \(a\).
    \item Cube of side lengths \(b\).
    \item Box of dimensions \(\boxDim{3a}{b}{b}\).
    \item Box of dimensions \(\boxDim{3b}{a}{a}\).
    \end{itemize}
    
    & 
    
    \begin{itemize}
    \item Box of dimensions \(\boxDim{6}{a}{a}\).
    \item Box of dimensions \(\boxDim{6}{b}{b}\).
    \item Box of dimensions \(\boxDim{12}{a}{b}\).
    \item Box of dimensions \(\boxDim{1}{1}{a}\).
    \item Box of dimensions \(\boxDim{1}{1}{b}\).
    \item Cube of side lengths \(2^{1/3}\). 
    \end{itemize}
\end{tabular}

Cardano's idea is to use some or all of the non-cube volumes on the left to cancel problematic terms on the right.
This is done in two stages:
\begin{enumerate}
\item (Depress the Cubic) Divide \(x = a + b\), and use \(a,b\) to remove the second order terms from the equation.
\item Say the new free variable is \(a\), then divide \(a = c + d\) and use \(c,d\) to remove the first order terms. 
\end{enumerate} 
We will look at following these ideas using algebraic manipulation.

\subsubsection*{The Problem}
Use Cardano's ideas to solve 
\begin{equation}
x^3 = 6x^2 + 3x + 2.
\end{equation}

\subsubsection*{The Solution}

We follow the two stages of Cardano.

\begin{enumerate}
\item Fist we depress the cubic. We split \(x = a + b\). So we get
\begin{equation}
a^3 + 3a^2b + 3 ab^2 + b^3 = 6a^2 + 12 ab + 6b^2 + 3a + 3b + 2.
\end{equation}
Now, note that if we fix \(b\) and let \(a\) be a new free variable, then it is possible to eliminate all
terms for \(a^2\). That is we need to find \(b\) such that \(3a^2b = 6a^2\). So we choose \(b = 2\). Then
we get
\begin{equation}
a^3 + 12a + 8 = 24a + 24 + 3a + 8.
\end{equation}
Collecting all non-third order terms to the right, we get
\begin{equation}
a^3 = 15a + 24.
\end{equation}

\item Now we split \(a = c + d\) to get
\begin{equation}
c^3 + 3c^2d + 3cd^2 + d^3 = 15c + 15d + 24. 
\end{equation}
Now, note that to get rid of first order terms and not introduce second order terms, we must use the entirety of the
non-cube terms on the left \(3c^2d + 3cd^2\) and eliminate all of the linear terms on the right \(15c + 15d\).
So we look at the possibilities of matching
\begin{equation}
3c^2d + 3cd^2 = 15(c + d).
\end{equation}
Now, note that \(3c^2d + 3cd^2 = 3cd(c + d)\). So, we are lead to \(3cd = 15\). Now after eliminating the matching, we
have the following system
\begin{equation}
\begin{cases}
c^3 + d^3 = 24, \\
cd = 5.
\end{cases}
\end{equation}

From this, we get that
\begin{equation}
125 + d^6 = 24d^3.
\end{equation}
This is quadratic in \(d^3\), so we may solve for \(d^3\) using the quadratic formula. We get
\begin{align}
d^3 & = \frac{24 \pm \sqrt{24^2 - 4 * 125}}{2}, \\ 
& = 12 \pm \sqrt{144 - 125}, \\ 
& = 12 \pm \sqrt{19}.
\end{align} 
Before, we continue, note that
\begin{align}
\frac{1}{12 + \sqrt{19}} & = \frac{12 - \sqrt{19}}{144 - 19}, \\
    & = \frac{12 - \sqrt{19}}{125}. 
\end{align}

Therefore, from \(cd = 5\), we see that when \(c = \left(12 + \sqrt{19}\right)^{1/3}\), we have that 
\(d = \left(12 - \sqrt{19}\right)^{1/3}\). The opposite situation holds, but without a loss of generality
we may suppose that the magnitudes \(|c|\) and \(|d|\) are given by the above.

Finally, we deal with the fact that there are three cube roots in the complex numbers. Let \(\zeta = e^{i2\pi / 3}\).
From \(cd = 5\), we have
\begin{equation}
c = \left(12 + \sqrt{19}\right)^{1/3} \zeta^k, d = \left(12 - \sqrt{19}\right)^{1/3} \zeta^{-k},
\end{equation}
for \(k = 0, 1, 2\).

Finally, we use that \(a = c + d\) to get that
\begin{equation}
a = \left(12 + \sqrt{19}\right)^{1/3} \zeta^k + \left(12 - \sqrt{19}\right)^{1/3} \zeta^{-k},
\end{equation}
for \(k = 0, 1, 2\). 

Then we use \(x = 2 + a\) to get
\begin{equation}
x = 2 + \left(12 + \sqrt{19}\right)^{1/3} \zeta^k + \left(12 - \sqrt{19}\right)^{1/3} \zeta^{-k},
\end{equation}
for \(k = 0, 1, 2\). We have found the three roots of the equation. 

\end{enumerate}
